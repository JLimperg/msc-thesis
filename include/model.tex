\chapter{Reflexive Graph Model}
\label{sec:model}

I now present a reflexive graph model of λST. The model is very similar to the
standard set-theoretic models of (dependent) type theories, but it additionally
captures a parametricity property: When we interpret a term $f ∶ \All{n}{T}$,
we get a dependent function $⟦f⟧ ∶ (n ∶ \Size) → ⟦T⟧(n)$, but this
function is parametric in the size -- i.e., the result of $⟦f⟧$ is independent,
in an appropriate sense, of which $n$ we apply it to.

To develop this model, we first introduce its central structure, (propositional)
reflexive graphs and their families. We then show how to interpret each
construct of λST -- sizes, types and terms.


\section{Propositional Reflexive Graphs}
\label{sec:model:rgraph}

This section introduces the categories of reflexive graphs, which will be used
to interpret sizes and size contexts. The results are formalised in
\icode{Model.\allowbreak RGraph}.

A \Def{reflexive graph} is a tuple $(Δ, ≈_Δ, \refl_Δ)$ where
\begin{AlignAnnot*}
  Δ &∶& \Type \\
  ≈_Δ &∶& Δ → Δ → \Type \\
  \refl_Δ(δ) &∶& δ ≈_Δ δ &\quad ∀ δ ∶ Δ.
\end{AlignAnnot*}
We will identify a reflexive graph with its underlying type where this is
unambiguous.

The relation $≈_Δ$ is what distinguishes this model from the standard
set-theoretical model of type theory. As we interpret the constructions of λST
in the next sections, we will define $≈_Δ$ to be, roughly speaking,
\enquote*{equality up to sizes}. For example, the interpretation $⟦Δ⟧$ of a size
context $Δ$ will have $δ ≈_{⟦Δ⟧} δ′$ for all $δ$ and $δ′$ because any two
interpretations of a size context should be considered equal up to sizes.

\begin{remark}
  Each reflexive graph is isomorphic to a presheaf over a particular category.
  Thus, reflexive graphs form a category with families \cite{dybjer1995}, i.e. a
  model of Martin-Löf type theory. It is thus not surprising that λST also
  admits a reflexive graph model. Moreover, many of the following results are
  direct consequences of the theory of presheaves, but to keep the presentation
  elementary, I prefer to make the constructions explicit.
\end{remark}

A reflexive graph $Δ$ is \Def{propositional} if $≈_Δ$ is a proposition (i.e. for
any $δ, δ′ ∶ Δ$ and $p, q ∶ δ ≈_Δ δ′$, $p = q$) and the underlying type $Δ$ is a
set (i.e. equality on $Δ$ is a proposition). We call such a graph a
\Def{PRGraph}. These conditions are trivially fulfilled for all the PRGraphs we
consider, so I will generally omit their proofs.

PRGraphs are simpler than general reflexive graphs because the proofs
$\refl_Δ(δ)$ are irrelevant. This saves us a lot of effort, particularly in the
formalisation. However, it means that the model probably would not accommodate
theories where the identity type is non-propositional, such as cubical type
theories.

Homomorphisms of PRGraphs are relation-preserving functions. More explicitly, a
\Def{PRGraph morphism} between PRGraphs $Δ$ and $Ω$ is a function $σ$ between
their underlying types such that
\begin{displaymath}
  δ ≈_Δ δ′ ⇒ σ(δ) ≈_Ω σ(δ′) \quad ∀ δ, δ′ ∶ Δ.
\end{displaymath}
Given a proof $p ∶ δ ≈_Δ δ′$, I will occasionally write $σ(p) ∶ σ(δ) ≈_Ω σ(δ′)$.
Not very often, though, since by propositionality of $Ω$, it does not matter
which specific proof of $σ(δ) ≈_Ω σ(δ′)$ we consider. This also means that in
particular $σ(\refl_Δ(δ)) = \refl_Ω(σ(δ))$; for non-propositional RGraphs, we
would have to add this condition to the definition of morphisms.

PRGraphs and their morphisms form a category in the obvious way. We call this
category $\mathit{PRGraphs}$.


\section{Families of Propositional Reflexive Graphs}
\label{sec:model:rgraphfam}

To interpret types and type contexts, which are indexed by some size context, we
need to move from PRGraphs to \emph{families} of PRGraphs. These may be thought
of as PRGraphs indexed by PRGraphs. The results in this section are formalised
in \icode{Model.\allowbreak Type.\allowbreak Core}.

A \Def{family of PRGraphs} over a given PRGraph $Δ$ is a tuple $(T, ≈_{T,p},
\refl_T)$ where
\begin{AlignAnnot*}
  T &∶& Δ → \Type \\
  ≈_{T,p} &∶& T(δ) → T(δ′) → \Type &\quad ∀ p ∶ δ ≈_Δ δ′ \\
  \refl_T(x) &∶& x ≈_{T,\refl_Δ(δ)} x &\quad ∀ x ∶ T(δ)
\end{AlignAnnot*}
such that $T(δ)$ is a set for any $δ$ and $≈_{T,p}$ is a proposition for any $T$
and $p$.

Note that our \enquote*{equality up to sizes} is now heterogeneous: We can ask
whether an inhabitant of $T(δ)$ is equal to an inhabitant of $T(δ′)$, as long as
$δ$ and $δ′$ are equal up to sizes. In the model, $δ$ and $δ′$ will be
(interpretations of) size valuations, which are always equal up to sizes. We
will thus be able to compare, for example, terms of $\Nat{n}$ and $\Nat{m}$ for
arbitrary $n$ and $m$. This is necessary as well: If λST is indeed
size-irrelevant and we take a term $t ∶ \All{n}{\Nat{v_0}}$, we should be able
to conclude that $\sapp{t}{m} ≈ \sapp{t}{m′}$ even though these terms have types
$\Nat{m}$ and $\Nat{m′}$ respectively.

Morphisms of PRGraph families are again relation-preserving functions; we just
need to carry around some more indices. To be precise, a \Def{morphism between
  PRGraph families $T$ and $U$} over the PRGraph $Δ$ is a family
of functions
\begin{displaymath}
  f_δ ∶ T(δ) → U(δ) \quad ∀ δ ∶ Δ
\end{displaymath}
such that
\begin{displaymath}
  x ≈_{T,p} y ⇒ f(x) ≈_{U,p} f(y) \quad ∀ p ∶ δ ≈_Δ δ′;\; x ∶ T(δ);\; y ∶ T(δ′).
\end{displaymath}

PRGraph families and their morphisms again form an obvious category which we
call $\mathit{PRGraphFams}$.

The last basic construction on PRGraph families we will use in the model is the
analogue of size substitution. Size substitutions $σ ∶ Δ ⇒ Ω$ will be modelled
as PRGraph morphisms $⟦σ⟧ ∶ ⟦Δ⟧ → ⟦Ω⟧$, so we must say what it means to apply
such a morphism to a family of PRGraphs over $⟦Ω⟧$. To that end, we define
model-level substitution: Given a PRGraph morphism $σ ∶ Δ → Ω$ and a family of
PRGraphs $T$ over $Ω$, the \Def{application of $σ$ to $T$}, $\sub{T}{σ}$, is the
following PRGraph family over $Δ$:
\begin{Align*}
  (\sub{T}{σ})(δ) &≔& T(σ(δ)) \\
  \mathord{≈_{\sub{T}{σ},p}} &≔& \mathord{≈_{T,σ(p)}} \\
  \refl_{\sub{T}{σ}}(x) &≔& \refl_T(x)
\end{Align*}
The last equation is not immediately well-typed: $\refl_T(x)$ proves $x
≈_{T,\refl_Δ(σ(δ))} x$ (assuming $x ∶ T(σ(δ))$) whereas we need to prove $x
≈_{T,σ(\refl_Δ(δ))} x$. However, since $≈_Δ$ is propositional, $\refl_Δ(σ(δ))$
and $σ(\refl_Δ(δ))$ are, in fact, equal.


\section{Properties of PRGraphFams}
\label{sec:model:prgraphfam-properties}

As usual, we will interpret the contexts of λST as iterated products of types
and the function space as an exponential. We therefore prove that $\PRGraphFams$
admits these constructions. Throughout this section, let $Δ$ be an arbitrary
PRGraph.

\subsection{Finite Products}
\label{sec:model:product}

The following results are formalised in \icode{Model.\allowbreak Terminal} and
\icode{Model.\allowbreak Product}.

Let $T, U$ be two RGraph families over $Δ$. The \Def{product of $T$ and $U$}, $T
× U$, is defined pointwise. Explicitly, it is the following RGraph family:
\begin{Align*}
  (T × U)(δ) &≔& T(δ) × U(δ) \\
  (x, x′) ≈_{T×U,p} (y, y′) &≔& x ≈_{T,p} y ∧ x′ ≈_{U,p} y′ \\
  \refl_{T×U}(x, y) &≔& (\refl_T(x), \refl_U(y)).
\end{Align*}
In the first equation, the $×$ on the right-hand side is the usual cartesian
product of types. The projections out of $T × U$, $π₁$ and $π₂$, are the
projections out of $T ⊗ U$. It is easy to check that this defines a product in
$\PRGraphFams$.

To interpret all finite products, we additionally need a terminal object. The
\Def{terminal object of $\mathit{PRGraphFams}$} (\icode{Model.\allowbreak
  Terminal.\allowbreak hasTerminal}) is the PRGraph family $⊤$ with
\begin{Align*}
  ⊤(δ) &≔& ⊤ \\
  x ≈_{⊤,p} y &≔& ⊤
\end{Align*}
On the right-hand sides of the equations, $⊤$ is the unit type. The relation
$≈_⊤$ is trivial and so obviously reflexive.

\subsection{Exponentials}
\label{sec:model:exponential}

The following results are formalised in \icode{Model.\allowbreak Exponential}.

Before we can construct exponentials, we need an auxiliary definition: Given a
family of PRGraphs $T$ over $Δ$ and an object $δ$ of $Δ$, the \Def{application
  of $T$ to $δ$}, $\Ap(T, δ)$, is the following PRGraph:
\begin{Align*}
  \Ap(T, δ) &≔& T(δ) \\
  ≈_{\Ap(T, δ)} &≔& \mathord{≈_{T,\refl_Δ(δ)}} \\
  \refl_{\Ap(T, δ)}(x) &≔& \refl_T(x).
\end{Align*}

Using $\Ap$, we can define exponentials. Let $T$ and $U$ be PRGraph families
over $Δ$. The \Def{exponential of $T$ and $U$}, $T ↝ U$, is the following
PRGraph family:
\begin{AlignAnnot*}
  (T ↝ U)(δ) &≔& \Ap(T, δ) → \Ap(U, δ) \\
  f ≈_{T↝U,p} g &≔& ∀ x, y.\; x ≈_{T,p} y ⇒ f(x) ≈_{U,p} g(y) & \qquad (p ∶ δ ≈_Δ δ′)
\end{AlignAnnot*}
$\Ap(T, δ) → \Ap(U, δ)$ is the type of PRGraph morphisms from $\Ap(T, δ)$ to
$\Ap(U, δ)$. The relation of $T ↝ U$ is reflexive because for any such RGraph
morphism $f$, $x ≈_{T,\refl_Δ(δ)} y$ implies $f(x) ≈_{U,\refl_Δ(δ)} f(y)$.

To prove that $T ↝ U$ is indeed an exponential, we still need currying and
evaluation morphisms. Given a morphism of PRGraph families $f ∶ T × U → V$, we
define
\begin{gather*}
  \curry(f) ∶ T → U ↝ V \\
  \curry(f)_δ(t) ≔ \fun{u}{f(t, u)}.
\end{gather*}
It is easy to check that $\curry(f)_δ(t)$ is a PRGraph morphism from
$\Ap(U, δ)$ to $\Ap(V, δ)$ and that $\curry(f)$ is a morphism of PRGraph
families. We also define an evaluation morphism:
\begin{gather*}
  \eval ∶ (T ↝ U) × T → U \\
  \eval(f, t) ≔ f(t).
\end{gather*}
Again, it is easy to check that $\eval$ is a morphism of PRGraph families and
that $T ↝ U$ together with $\curry$ and $\eval$ is an exponential in
$\PRGraphFams(Δ)$.


\section{Sizes}
\label{sec:model:sizes}

With the central notions defined, we can proceed to model λST. We start with
sizes, whose interpretation is formalised in \icode{Model.\allowbreak Size}.

A (model-level) \Def{size} is either a natural number or $∞$ or $⋆$. We define a
strict order $<$ on sizes by
\begin{displaymath}
  0 < 1 < \dots < ∞ < ⋆.
\end{displaymath}
The reflexive closure of $<$ is called $≤$. We write $\Size$ for the type of
sizes and $\Size_{<n}$ for the type of sizes $m$ such that $m < n$.

The orders $<$ and $≤$ are straightforward \enquote*{extensions} of $<_ℕ$ and
$≤_ℕ$ (the usual orders on natural numbers) and so largely behave as expected.
In particular, we will need the following properties, which are all easy to
prove.

\begin{lemma}[Properties of $<$ and $≤$]
  \label{lem:<-props}
  For arbitrary sizes $n$, $m$, $o$, we have:
  \begin{enumerate}
    \item $<$ and $≤$ are transitive.
    \item If $n ≤ m$ and $m < o$ then $n < o$. If $n < m$ and $m ≤ o$ then $n < o$.
    \item If $n, m ∈ ℕ$ and $n ≤_ℕ m$ then $n ≤ m$.
    \item $0 ≤ n$.
    \item If $n ∈ ℕ$ and $n < m$ then $n + 1 ≤ m$.
    \item If $n ∈ ℕ$ and $n + 1 ≤ m$ then $n < m$.
    \item $<$ is irreflexive, $≤$ is antisymmetric.
    \item $<$ is well-founded, i.e. there is no infinite descending chain $n₀ >
      n₁ > \dots$.
  \end{enumerate}
\end{lemma}

To interpret the successor of a size $n$, $\ssuc{n}$, we need an appropriate
model-level successor. We thus define the \Def{successor} of a size $n$ as
\begin{displaymath}
  \mssuc(n) ≔
    \begin{cases}
      n + 1 & \qquad \text{if $n ∈ ℕ$} \\
      n & \qquad \text{otherwise}
    \end{cases}
\end{displaymath}
This successor is not quite well-behaved with respect to the strict order: we
have $\mssuc(∞) = ∞$ and thus not necessarily $n < \mssuc(n)$. However, we do
have $n < \mssuc(n)$ if $n < ∞$ (which is equivalent to $n ∈ ℕ$), so we can
interpret the successor rule of the syntactic $<$ relation. Our size successor
also preserves the relations $<$ and $≤$.

This is all the infrastructure we need to model sizes. A size context $Δ$ is
interpreted as a type. A well-typed size in $Δ$ is interpreted as a function
from $Δ$ to $\Size$. These interpretations are mutually recursively defined:
\begin{AlignAnnot*}
  ⟦()⟧ &≔& ⊤ &\qquad \text{($⊤$ is the unit type)} \\
  ⟦\ctx{Δ}{n}⟧ &≔& \Sigma_{δ ∶ ⟦Δ⟧}\Size_{<⟦n⟧(δ)} \\
  \\
  ⟦v_0⟧(δ, n) &≔& n \\
  ⟦v_{x+1}⟧(δ, n) &≔& ⟦v_x⟧(δ) \\
  ⟦0⟧(δ) &≔& 0 \\
  ⟦\ssuc{n}⟧(δ) &≔& \mssuc(⟦n⟧(δ)) \\
  ⟦∞⟧(δ) &≔& ∞ \\
  ⟦⋆⟧(δ) &≔& ⋆ \\
\end{AlignAnnot*}

We will also consider $⟦Δ⟧$ as a PRGraph whose underlying type is $⟦Δ⟧$ and
whose relation is trivial, i.e. any two elements are related. A size
interpretation $⟦n⟧$ is then a PRGraph morphism (if we similarly treat the
type $\Size$ as a trivial PRGraph).

\begin{remark}
  We will shortly interpret a type $T$ in a context $Δ$ as a family of PRGraphs
  indexed by the PRGraph $⟦Δ⟧$. However, since $⟦Δ⟧$ is always trivial as a
  PRGraph, we could also interpret $T$ as a family of PRGraphs indexed by the
  \emph{type} $⟦Δ⟧$, which would be a little simpler. I choose not to do so
  because that model would likely not support dependent types (and perhaps other
  interesting features).
\end{remark}

An important property of our interpretation of sizes is that the syntactic $<$
relation implies the semantic one. To prove this, we need a helper lemma about
weakening.

\begin{lemma}[Interpretation of $\wk$]
  \label{lem:⟦wk⟧}
  Assume $Δ ⊢ n$, $Δ ⊢ m$ and $(δ, o) ∶ ⟦\ctx{Δ}{n}⟧$. Then
  $⟦\wk(m)⟧(δ, o) = ⟦m⟧(δ)$.
\end{lemma}

\begin{lemma}[Interpretation of $<$]
  \label{lem:⟦<⟧}
  Assume $Δ ⊢ n < m$ and $δ ∶ ⟦Δ⟧$. Then $⟦n⟧ δ < ⟦m⟧ δ$.
\end{lemma}

\begin{proof}
  By induction on the derivation of $Δ ⊢ n < m$. In the variable case, we need
  to use Lemma \ref{lem:⟦wk⟧}. The other cases follow either directly from the
  definition of $<$ or from Lemma \ref{lem:<-props}.
\end{proof}

This concludes the interpretation of sizes. Next, we interpret size
substitutions as functions between the interpretations of size contexts. For
reasons explained in \secref{formalisation:sub}, the results presented here are
not exactly those formalised.

Let $σ ∶ Δ ⇒ Ω$ be a well-typed substitution. The interpretation of $σ$ is a
function from $⟦Δ⟧$ to $⟦Ω⟧$ defined by recursion on the derivation of $σ ∶ Δ ⇒
Ω$:
\begin{gather*}
  ⟦()⟧ ∶ ⟦Δ⟧ → ⟦()⟧ \\
  ⟦()⟧(δ) ≔ \mathord{tt} \qquad \text{($\mathord{tt}$ is the unique value of $⊤$)} \\
  \\
  ⟦\ssub{σ}{n}⟧ ∶ ⟦Δ⟧ → \Sigma_{ω ∶ ⟦Ω⟧}\Size_{<⟦m⟧(ω)} \\
  ⟦\ssub{σ}{n}⟧(δ) ≔ ⟦σ⟧(δ), n
\end{gather*}
The last equation is well-typed because $\ssub{σ}{n} ∶ Δ ⇒ \ctx{Ω}{m}$ implies
$Δ ⊢ n < \sub{m}{σ}$ and thus by Lemma \ref{lem:⟦<⟧} $⟦n⟧ < ⟦\sub{m}{σ}⟧$. The
following Lemma \ref{lem:⟦sub⟧} then lets us conclude $⟦n⟧ < ⟦m⟧(⟦σ⟧(δ))$. (This
means that the present definition and Lemma \ref{lem:⟦sub⟧} are in fact mutually
recursive.)

\begin{lemma}[Interpretation of size substitutions is correct]
  \label{lem:⟦sub⟧}
  If $σ ∶ Δ ⇒ Ω$ and $Ω ⊢ n$ then $⟦\sub{σ}{n}⟧ = ⟦n⟧ ∘ ⟦σ⟧$.
\end{lemma}

We will also need to know the interpretations of some of the specific
substitutions from \secref{source:sub}. The following lemma tells us what they
do when applied to an inhabitant of a size context. (TODO words)

\begin{lemma}[Interpretation of specific size substitutions]
  \label{lem:sub-app}
  Assume $(ω, m, k) ∶ ⟦\ctx{\ctx{Ω}{n}}{v_0}⟧$, $σ ∶ Δ → Ω$ and $τ ∶ Ω → Ω′$. Then we have
  \begin{AlignAnnot*}
    ⟦\Id⟧(ω) &=& ω \\
    ⟦σ \fcomp τ⟧(ω) &=& ⟦τ⟧(⟦σ⟧(ω)) \\
    ⟦\Wk⟧(ω, m) &=& ω \\
    ⟦\Fill(o)⟧(ω) &=& (ω, ⟦o⟧(ω)) & \quad (o < ⟦n⟧(ω)) \\
    ⟦\Lift(σ)⟧(ω, m) &=& (⟦σ⟧(ω), m) \\
    ⟦\Skip⟧(ω, m, k) &=& (ω, k)
  \end{AlignAnnot*}
\end{lemma}


\section{Types}
\label{sec:model:types}

In this section, we model the types of λST. The following results are formalised
in \icode{Model.\allowbreak Nat}, \icode{Model.\allowbreak Stream} and
\icode{Model.\allowbreak Quantification}.

As foreshadowed, we interpret well-scoped types $T$ in a size context $Δ$ as
families of PRGraphs $⟦T⟧$ over the PRGraph $⟦Δ⟧$. The interpretations of sized
naturals, sized streams and functions are straightforward; only size
quantification will be slightly more complex.

Let $ℕ_{≤n}$ be the type of natural numbers $m$ such that $m ≤ n$ (with $n$
an arbitrary size). The interpretation of $Δ ⊢ \Nat{n}$ is the following
family of PRGraphs:
\begin{Align*}
  ⟦\Nat{n}⟧(δ) &≔& ℕ_{≤ ⟦n⟧(δ)} \\
  i ≈_{⟦\Nat{n}⟧,p} j &≔& i = j.
\end{Align*}

Streams are interpreted in a similar fashion. The interpretation of $Δ ⊢
\Stream{n}$ is the following family of PRGraphs:
\begin{Align*}
  ⟦\Stream{n}⟧(δ) &≔& ℕ_{≤ ⟦n⟧(δ)} → ℕ \\
  f ≈_{⟦\Stream{n}⟧,p} g &≔& f = g.
\end{Align*}
Here, $=$ means extensional equality of functions.

The function space is interpreted by the exponential of PRGraph families from
\secref{model:exponential}:
\begin{displaymath}
  ⟦T → U⟧ ≔ ⟦T⟧ ↝ ⟦U⟧.
\end{displaymath}

Size quantification requires a little more effort. We want to think of the terms
of $\All{n}{T}$ as functions which take a size argument, but they should be
parametric in that argument: When we apply such a function to different sizes,
it should return results that are equal up to sizes. More precisely: Let $T$ be
a family of PRGraphs over $⟦\ctx{Δ}{n}⟧$ and $δ ∶ ⟦Δ⟧$. A function
\begin{displaymath}
  f ∶ (m ∶ \Size_{<⟦n⟧(δ)}) → T(δ, m)
\end{displaymath}
is \Def{size-parametric} if
\begin{displaymath}
  f(m) ≈_T f(m′) \quad ∀ m, m′ ∶ \Size_{<⟦n⟧(δ)}
\end{displaymath}
This equation is well-typed because $≈_{⟦\ctx{Δ}{n}⟧}$ is trivial, so in
particular $(δ, m) ≈_{\ctx{Δ}{n}} (δ, m′)$. We write $\Param(T,δ)$ for the type
of size-parametric functions into $T$.

Now we can interpret size quantification: The interpretation of $Δ ⊢ \All{n}{T}$
is the PRGraph family $Π(⟦n⟧, ⟦T⟧)$ with
\begin{AlignAnnot*}
  Π(n, T)(δ) &≔& \Param(T,δ) \\
  f ≈_{Π(n, T)} g &≔& ∀ m, m′.\; f(m) ≈_T g(m′).
\end{AlignAnnot*}
If we think of $≈_T$ as equality up to sizes, $⟦\All{n}{T}⟧$ reflects
the intuition that when we apply a term of $\All{n}{T}$ to a size $m$, it does
not matter -- up to sizes -- which $m$ we choose.

Having interpreted all types, we can now interpret type contexts. These are mere
lists of types, so they can be modelled by finite products. The interpretation
of $Δ ⊢ Γ$ is defined by recursion over the derivation:
\begin{Align*}
  ⟦()⟧ &≔& ⊤ \\
  ⟦\Ctx{Γ}{T}⟧ &≔& ⟦Γ⟧ × ⟦T⟧.
\end{Align*}
$⊤$ is the terminal PRGraph family and $×$ is the product of PRGraph families,
both defined in \secref{model:product}.

With this, we have interpreted all type-level constructs of λST. What remains is
to prove a substitution lemma stating that syntax-level and model-level
substitution agree. To that end, we need some additional lemmas about
model-level substitution.

\begin{lemma}[Substitution in exponentials]
  Let $T$, $U$ be PRGraph families over $Ω$ and $σ ∶ Δ → Ω$ a PRGraph morphism. Then
  \begin{displaymath}
    \sub{T ↝ U}{σ} = \sub{T}{σ} ↝ \sub{U}{σ}.
  \end{displaymath}
\end{lemma}

(TODO maybe prove this)

\begin{remark}
  The appropriate notion of equality for PRGraph families is isomorphism in
  $\PRGraphFams(Δ)$, so the $=$ above really means $≅$. This abuse of notation
  is justified because in the formalisation, we can prove that isomorphic
  PRGraph families are in fact propositionally equal.
\end{remark}

\begin{lemma}[Substitution in quantifications]
  Let $T$ be a PRGraph family over $⟦\ctx{Ω}{n}⟧$ and $σ ∶ Δ ⇒ Ω$ a well-typed
  substitution. Then
  \begin{displaymath}
    \sub{Π(n, T)}{⟦σ⟧} = Π(\sub{n}{⟦σ⟧}, \sub{T}{⟦\Lift(σ)⟧}).
  \end{displaymath}
\end{lemma}

(TODO maybe prove this)

From the previous two lemmas follows the main substitution lemma for types and
contexts.

\begin{lemma}[Interpretation of substitution in types and contexts is correct]
  \label{lem:⟦subT⟧}
  If $Ω ⊢ T$, $Ω ⊢ Γ$ and $σ ∶ Δ ⇒ Ω$ then
  \begin{Align*}
    ⟦\sub{T}{σ}⟧ &=& \sub{⟦T⟧}{⟦σ⟧},  \\
    ⟦\sub{Γ}{σ}⟧ &=& \sub{⟦Γ⟧}{⟦σ⟧}.
  \end{Align*}
\end{lemma}


\section{Terms}
\label{sec:model:terms}

Well-typed terms (or, more precisely, judgments $Δ;Γ ⊢ t ∶ T$) are interpreted
as morphisms between the PRGraph families $⟦Γ⟧$ and $⟦T⟧$. In this section, we
show a natural interpretation of each of the terms of λST. Formalisations of
these definitions can be found in \icode{Model.\allowbreak Term}.

\subsection{Functions}
\label{sec:model:terms:functions}

Recall that the function space is modelled by the exponential $⟦T⟧ ↝ ⟦U⟧$ as in
the standard categorical model of the simply-typed lambda calculus. Accordingly,
the interpretations of its constructor, abstraction, and eliminator,
application, are also standard.

To interpret $\lam{T}{t}$, we may assume a morphism $⟦t⟧ ∶ ⟦Γ⟧ × ⟦T⟧ → ⟦U⟧$. We
then define
\begin{Align*}
  ⟦\lam{T}{t}⟧ &∶& ⟦Γ⟧ → ⟦T⟧ ↝ ⟦U⟧ \\
  ⟦\lam{T}{t}⟧ &≔& \curry(⟦t⟧).
\end{Align*}

The interpretation of $\app{t}{u}$, assuming $⟦t⟧ ∶ ⟦Γ⟧ → ⟦T ↝ U⟧$ and $⟦u⟧ ∶ ⟦Γ⟧ →
⟦T⟧$, is
\begin{Align*}
  ⟦\app{t}{u}⟧ &∶& ⟦Γ⟧ → ⟦U⟧ \\
  ⟦\app{t}{u}⟧ &≔& \eval ∘ \anglebrackets{⟦t⟧ × ⟦u⟧}.
\end{Align*}
The operator
\begin{displaymath}
  ⟨∙×∙⟩ ∶ (A → A′) → (B → B′) → A × B → A′ × B′
\end{displaymath}
can be defined in any category with products.


\subsection{Size Quantification}
\label{sec:model:terms:quantification}

To interpret size quantification, we must consider size abstractions and size
applications. They are formalised in \icode{Model.\allowbreak Quantification}.
Their typing rules are reproduced in \figref{typing:quantification} for
reference.

\begin{figure}
  \begin{mathpar}
    \inferrule{\ctx{Δ}{n}; \sub{Γ}{\Wk} ⊢ t ∶ T \\ Δ ⊢ Γ}
      {Δ; Γ ⊢ \slam{n}{t} ∶ \All{n}{T}}

    \inferrule{Δ; Γ ⊢ t ∶ \All{n}{T} \\ Δ ⊢ m < n}
      {Δ; Γ ⊢ \sapp{t}{m} ∶ \sub{T}{\Fill(m)}}
  \end{mathpar}

  \caption{Typing rules for terms related to size quantification}
  \label{fig:typing:quantification}
\end{figure}

When interpreting size quantification, we may assume a morphism $⟦t⟧ ∶
⟦\sub{Γ}{\Wk}⟧ → ⟦T⟧$ with components
\begin{AlignAnnot*}
  ⟦t⟧_{(δ,n)} &∶& ⟦\sub{Γ}{\Wk}⟧(δ, n) → ⟦T⟧(δ, n) \\
    &=& \sub{⟦Γ⟧}{⟦\Wk⟧}(δ, n) → ⟦T⟧(δ, n) &\quad \text{(by Lemma \ref{lem:⟦subT⟧})} \\
    &=& ⟦Γ⟧(⟦\Wk⟧(δ, n)) → ⟦T⟧(δ, n) &\quad \text{(by definition)} \\
    &=& ⟦Γ⟧(δ) → ⟦T⟧(δ, n) &\quad \text{(by Lemma \ref{lem:sub-app}).}
\end{AlignAnnot*}
We then define
\begin{gather*}
  ⟦\slam{n}{t}⟧ ∶ ⟦Γ⟧ → ⟦\All{n}{T}⟧ \\
  ⟦\slam{n}{t}⟧_δ(γ) ≔ \fun{m < ⟦n⟧(δ)}{⟦t⟧_{(δ,m)}(γ)}.
\end{gather*}
This definition is valid due to the following observations:
\begin{itemize}
\item The right-hand side of the equation is a size-parametric function.
  Recall that $(δ, m) ≈_{⟦\ctx{Δ}{n}⟧} (δ, m′)$ for all $m, m′$. We also have
  $γ ≈_{⟦Γ⟧} γ$ by reflexivity and this implies $⟦t⟧_{(δ, m)}(γ) ≈_{⟦T⟧}
  ⟦t⟧_{(δ, m′)}(γ)$ because $⟦t⟧$ is a morphism of PRGraph families.
\item $⟦\slam{n}{t}⟧$ is a morphism of PRGraph families. This follows from
  essentially the same argument, only replacing the fact $γ ≈_{⟦Γ⟧} γ$ with an
  assumption $γ ≈_{⟦Γ⟧} γ′$ for some arbitrary $γ′$.
\end{itemize}

Moving to size application, we may assume a morphism
\begin{displaymath}
  ⟦t⟧ ∶ ⟦Γ⟧ → ⟦\All{n}{T}⟧.
\end{displaymath}
and we must construct a morphism with components
\begin{AlignAnnot*}
  ⟦\sapp{t}{m}⟧_δ &∶& ⟦Γ⟧(δ) → ⟦\sub{T}{\Fill(m)}⟧ \\
  &=& ⟦Γ⟧(δ) → ⟦T⟧(δ, ⟦m⟧(δ)) &\quad \text{(by Lemmas \ref{lem:⟦subT⟧}, \ref{lem:sub-app}).}
\end{AlignAnnot*}
That morphism is
\begin{displaymath}
  ⟦\sapp{t}{m}⟧_δ(γ) ≔ ⟦t⟧_δ(γ)(⟦m⟧(δ)).
\end{displaymath}
The size $⟦m⟧(δ)$ is a valid argument to $⟦t⟧_δ(γ)$ because we have $Δ ⊢ m < n$
and thus by Lemma \ref{lem:⟦<⟧} $⟦m⟧(δ) < ⟦n⟧(δ)$ for all $δ$. $⟦\sapp{t}{m}⟧$
is a morphism of PRGraph families because $⟦t⟧$ is.


\subsection{Natural Numbers}
\label{sec:model:terms:nat}

The sized natural number type $\Nat{n}$ has constructors zero and successor and
eliminator $\mathrm{caseNat}$. Their typing rules are reproduced in
\figref{typing:nat} for reference. Their interpretations are formalised in
\icode{Model.\allowbreak Stream}. Recall that $\Nat{n}$ is modelled by the type
of sizes below $n$.

\begin{figure}
  \begin{mathpar}
    \inferrule{Δ ⊢ n < ⋆ \\ Δ ⊢ Γ}{Δ; Γ ⊢ \zero{n} ∶ \Nat{n}}

    \inferrule{Δ ⊢ n < ⋆ \\ Δ ⊢ m < n \\ Δ; Γ ⊢ i ∶ \Nat{m}}
    {Δ; Γ ⊢ \suc{n}{m}{i} ∶ \Nat{n}}

    \inferrule{Δ ⊢ T \\ Δ ⊢ n < ⋆ \\ Δ; Γ ⊢ i ∶ \Nat{n} \\ Δ; Γ ⊢ z ∶ T \\
      Δ; Γ ⊢ s ∶ \All{n}{\Nat{v_0} → \sub{T}{\Wk}}}
    {Δ; Γ ⊢ \case{T}{n}{i}{z}{s} ∶ T}
  \end{mathpar}
  \caption{Typing rules for terms related to $\Nat{n}$}
  \label{fig:typing:nat}
\end{figure}

The term $\zero{n}$ is interpreted by
\begin{displaymath}
  ⟦\zero{n}⟧_δ(γ) ≔ 0.
\end{displaymath}
This is valid since $0 ≤ n$ for all $n$ (by Lemma \ref{lem:<-props}).

To interpret the successor, given a morphism $⟦i⟧ ∶ ⟦Γ⟧ → ⟦\Nat{n}⟧$, we define
\begin{displaymath}
  ⟦\suc{n}{m}{i}⟧_δ(γ) ≔ ⟦i⟧_δ(γ) + 1.
\end{displaymath}
This is valid because we have $⟦i⟧_δ(γ) ≤ ⟦m⟧(δ)$ (by definition of $⟦\Nat{m}⟧$)
and $⟦m⟧(δ) < ⟦n⟧(δ)$ (by Lemma \ref{lem:⟦<⟧}), which implies $⟦i⟧_δ(γ) + 1 ≤
⟦n⟧(δ)$.

To interpret the eliminator $\mathrm{caseNat}$, we assume morphisms $⟦i⟧ ∶ ⟦Γ⟧ →
⟦\Nat{n}⟧$, $⟦z⟧ ∶ ⟦Γ⟧ → ⟦T⟧$ and $⟦s⟧ ∶ ⟦Γ⟧ → ⟦\All{n}{\Nat{v_0} →
  \sub{T}{\Wk}}⟧$. The interpretation is then
\begin{displaymath}
  ⟦\case{T}{n}{i}{z}{s}⟧_δ(γ) ≔
    \begin{cases}
      ⟦z⟧_δ(γ) &\quad \text{if $⟦i⟧_δ(γ) = 0$} \\
      ⟦s⟧_δ(γ)(⟦n⟧ - 1)(⟦i⟧_δ(γ) - 1) &\quad \text{otherwise.}
    \end{cases}
\end{displaymath}
Note that in the second branch, $⟦n⟧(δ)$ cannot be zero because $⟦i⟧_δ(γ) ≤
⟦n⟧(δ)$ and $⟦i⟧_δ(γ) ≠ 0$.


\subsection{Streams}
\label{sec:model:terms:stream}

Sized streams are constructed by the term $\mathrm{cons}$ and eliminated by
$\mathrm{head}$ and $\mathrm{tail}$. The typing rules of these terms are
reproduced in \figref{typing:stream}. Their interpretations are formalised in
\icode{Model.\allowbreak Stream}. Recall that the model of $\Stream{n}$ is
functions from naturals below $n$ to naturals.

\begin{figure}
  \begin{mathpar}
    \inferrule{Δ ⊢ n < ⋆ \\ Δ; Γ ⊢ i ∶ \Nat{∞} \\
      Δ; Γ ⊢ \V{is} ∶ \All{n}{\Stream{v_0}}}
    {Δ; Γ ⊢ \cons{n}{i}{\V{is}} ∶ \Stream{n}}

    \inferrule{Δ ⊢ n < ⋆ \\ Δ; Γ ⊢ \V{is} ∶ \Stream{n}}
    {Δ; Γ ⊢ \head{n}{\V{is}} ∶ \Nat{∞}}

    \inferrule{Δ ⊢ n < ⋆ \\ Δ; Γ ⊢ \V{is} ∶ \Stream{n} \\ Δ ⊢ m < n}
    {Δ; Γ ⊢ \tail{n}{\V{is}}{m} ∶ \Stream{m}}
  \end{mathpar}
  \caption{Typing rules for terms related to $\Stream{n}$}
  \label{fig:typing:stream}
\end{figure}

To interpret $\mathrm{cons}$, we assume morphisms $⟦i⟧$ and $⟦\V{is}⟧$ with
components
\begin{Align*}
  ⟦i⟧_δ
  &∶& ⟦Γ⟧(δ) → ⟦\Nat{∞}⟧(δ) \\
  &=& ⟦Γ⟧(δ) → ℕ_{<∞} \\
  &=& ⟦Γ⟧(δ) → ℕ \\
  ⟦\V{is}⟧_δ
  &∶& ⟦Γ⟧(δ) → ⟦\All{n}{\Stream{v_0}}⟧(δ) \\
  &=& ⟦Γ⟧(δ) → \Param(⟦\Stream{v_0}⟧, δ) \\
  &≈& ⟦Γ⟧(δ) → (m ∶ \Size_{<⟦n⟧(δ)}) → ⟦\Stream{v_0}⟧(δ, n) \\
  &=& ⟦Γ⟧(δ) → (m ∶ \Size_{<⟦n⟧(δ)}) → \Size_{≤m} → ℕ.
\end{Align*}
The $≈$ on the penultimate line indicates that we are ignoring the parametricity
requirement of $\Param$ to unfold it. We then define
\begin{displaymath}
  ⟦\cons{n}{i}{\V{is}}⟧_δ(γ)(m) ≔
  \begin{cases}
    ⟦i⟧_δ(γ) &\quad \text{if $m = 0$} \\
    ⟦\V{is}⟧_δ(γ)(m - 1)(m - 1) &\quad \text{otherwise}.
  \end{cases}
\end{displaymath}
Note that $m ≤ ⟦n⟧(δ)$, so in the second equation we have $m - 1 < ⟦n⟧(δ)$ and
obviously $m - 1 ≤ m - 1$.

The destructors of $\Stream{n}$ have less complex interpretations. For
$\mathrm{head}$, we assume a morphism $⟦\V{is}⟧ ∶ ⟦\Stream{n}⟧$ and define
\begin{gather*}
  ⟦\head{n}{\V{is}}⟧ ∶ ⟦Γ⟧ → ⟦\Nat{∞}⟧ \\
  ⟦\head{n}{\V{is}}⟧_δ(γ) ≔ ⟦\V{is}⟧_δ(γ)(0).
\end{gather*}
Similarly, for $\mathrm{tail}$, we assume $⟦\V{is}⟧ ∶ ⟦\Stream{n}⟧$ and $Δ ⊢ m <
n$ and define
\begin{gather*}
  ⟦\tail{n}{\V{is}}{m}⟧ ∶ ⟦\Stream{m}⟧ \\
  ⟦\tail{n}{\V{is}}{m}⟧_δ(γ)(k) ≔ ⟦\V{is}⟧_δ(γ)(k + 1).
\end{gather*}
The application of $k + 1$ is well-typed because $k ≤ ⟦m⟧(δ)$ and $⟦m⟧(δ) <
⟦n⟧(δ)$ (by Lemma \ref{lem:⟦<⟧}).


\subsection{Fixpoint}
\label{sec:model:terms:fixpoint}

The sized fixpoint combinator is at the heart of λST, being the sole means of
recursion. As such, its interpretation is also recursive but otherwise
straightforward.

Recall the typing rule of $\mathrm{fix}$:
\begin{displaymath}
  \centering
  \inferrule{\ctx{Δ}{⋆} ⊢ T \\
    Δ; Γ ⊢ t ∶ \All{⋆}{(\All{v_0}{\sub{T}{\Skip}}) → T} \\ Δ ⊢ n < ⋆}
  {Δ; Γ ⊢ \fix{T}{t}{n} ∶ \sub{T}{\Fill(n)}}
\end{displaymath}
According to this rule, we may assume a morphism $⟦t⟧$ with components
\begin{Align*}
  ⟦t⟧_δ
    &∶& ⟦Γ⟧(δ) → ⟦\All{⋆}{(\All{v_0}{\sub{T}{\Skip}}) → T}⟧(δ) \\
    &=& ⟦Γ⟧(δ) → (n ∶ \Size_{<⋆}) → ((m ∶ \Size_{<n}) → ⟦T⟧(⟦\Skip⟧(δ, n, m))) →
    ⟦T⟧(δ, n) \\
    &=& ⟦Γ⟧(δ) → (n ∶ \Size_{<⋆}) → ((m ∶ \Size_{<n}) → ⟦T⟧(δ, m)) → ⟦T⟧(δ, n)
\end{Align*}
The first step in this chain uses the fact that $⟦\sub{T}{\Skip}⟧ =
\sub{⟦T⟧}{⟦\Skip⟧}$ (Lemma \ref{lem:⟦subT⟧}). In the second step, we simplify
$⟦\Skip⟧(δ, n, m)$ to $(δ, m)$ according to Lemma \ref{lem:sub-app}.

Given such a $⟦t⟧$, we first define the following auxiliary function:
\begin{gather*}
  f_δ ∶ ⟦Γ⟧(δ) → (n ∶ \Size_{<⋆}) → ⟦T⟧(δ, n) \\
  f_δ(γ)(n) ≔ ⟦t⟧_δ(γ)(n)(\fun{m}{f_δ(γ)(m)})
\end{gather*}
% TODO calling convention: f(x)(y) or f(x, y)?
This function is a standard fixpoint combinator. It terminates because the
recursive call is on a size $m < n$ and $<$ is well-founded: There is no
infinite descending chain $n₀ > n₁ > \dots$ and so a recursion which always
decreases the size argument must eventually stop. Interpreting $\mathrm{fix}$ is
then just a matter of applying $f$:
\begin{Align*}
  ⟦\fix{T}{t}{n}⟧_δ
    &∶& ⟦Γ⟧(δ) → ⟦\sub{T}{\Fill{n}}⟧(δ) \\
    &=& ⟦Γ⟧(δ) → ⟦T⟧(δ, ⟦n⟧(δ)) \\
  ⟦\fix{T}{t}{n}⟧_δ(γ) &≔& f_δ(γ)(⟦n⟧(δ)).
\end{Align*}

In the above argument, we have, as usual, glossed over various proof obligations
concerning size parametricity, preservation of relations and so on. However, one
of these obligations is actually nontrivial: In the definition of the auxiliary
function $f$, we use $f$ recursively as a size-parametric function. This means
that we must prove $f$ size-parametric \emph{mutually recursively} with its own
definition. The full, somewhat gory details of this construction appear in
\secref{formalisation:fixpoint}.

% TODO add conclusion of this section
