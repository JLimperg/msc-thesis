\chapter{Source Language}

\section{Source Language}

Our source language, λST, is a simply-typed lambda calculus extended with sizes,
bounded size quantification, sized natural numbers and a sized fixpoint
combinator.


\subsection{Syntax}

The syntax of λST is summarised in Fig. \ref{fig:syntax}.

\begin{figure}
  \begin{displaymath}
    \begin{array}{llcll}
      \text{size} & i, j, k & ∷= & α & \text{(size variable)} \\
      & & | & ↑ n & \text{(successor)} \\

      \text{extended size} & n, m, o & ∷= & i & \text{(size)} \\
      & & | & ∞ & \text{(infinity)} \\

      \text{size context} & Δ & ∷= & () & \text{(empty context)}  \\
      & & | & \ctx{Δ}{α<n} & \text{(context extension)} \\

      \text{type} & T, U, V & ∷= & ℕ~n & \text{(sized natural numbers)} \\
      & & | & T ⇒ U & \text{(functions)} \\
      & & | & \all{x<n}{T} & \text{(size quantification)} \\

      \text{type context} & Γ & ∷= & () & \text{(empty context)} \\
      & & | & \ctx{Γ}{t \ty T} & \text{(context extension)} \\

      \text{term} & t, u, v & ∷= & x & \text{(term variable)} \\
      & & | & \lam{x \ty T}{t} & \text{(abstraction)} \\
      & & | & t \ap u & \text{(application)} \\
      & & | & \lam{x < n}{t} & \text{(size abstraction)} \\
      & & | & t \ap n & \text{(size application)} \\
      & & | & 0 & \text{(zero)} \\
      & & | & \suc & \text{(successor)} \\
      & & | & \case[T] & \text{(eliminator for ℕ)} \\
      & & | & \fix[T] & \text{(sized fixpoint)} \\
    \end{array}
  \end{displaymath}

  \caption{Syntax of λST}
  \label{fig:syntax}
\end{figure}


\subsection{Typing}


\section{Model}
TODO


\section{Formalisation}
TODO
