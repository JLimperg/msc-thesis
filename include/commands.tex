% Code environment
\newenvironment{code}
  {\VerbatimEnvironment
   \begin{minted}{Agda}}
  {\end{minted}
   \par}


% Theorem environments

\theoremstyle{definition}
\newtheorem{definition}{Definition}[chapter]

\theoremstyle{plain}
\newtheorem{theorem}[definition]{Theorem}
\newtheorem{lemma}[definition]{Lemma}

\theoremstyle{remark}
\newtheorem*{remark}{Remark}


% crude replacement for align
\newenvironment{Align*}
{\begin{displaymath}\begin{array}{lcl}}
{\end{array}\end{displaymath}}

\newenvironment{AlignAnnot*}
{\begin{displaymath}\begin{array}{lcll}}
{\end{array}\end{displaymath}}


% Additional commands
\newcommand*{\secref}[1]{Sec.\ \ref{sec:#1}}
\newcommand*{\figref}[1]{Fig.\ \ref{fig:#1}}
\newcommand*{\chapref}[1]{Ch.\ \ref{sec:#1}}
\DeclarePairedDelimiter{\brackets}{[}{]}
\DeclarePairedDelimiter{\anglebrackets}{⟨}{⟩}

\newcommand*{\bnfdef}{\ensuremath{\Coloneqq}}
\newcommand*{\icode}[1]{\mbox{\texttt{#1}}}
\newcommand*{\Def}[1]{\emph{#1}}

\newcommand*{\ssuc}[1]{\ensuremath{\mathop{↑} #1}}
\newcommand*{\ctx}[2]{\ensuremath{#1,\, #2}}
\newcommand*{\ssub}[2]{\ensuremath{#1, #2}}
\newcommand*{\sub}[2]{\ensuremath{#1\brackets*{#2}}}
  % TODO The brackets* resizes the substitution brackets according to the size
  % of the object between them. Does this end up looking better or worse, on balance?
\newcommand*{\fcomp}{\ensuremath{\mathbin{\gg}}}

\newcommand*{\Nat}[1]{\ensuremath{\mathrm{Nat}~#1}}
\newcommand*{\Stream}[1]{\ensuremath{\mathrm{Stream}~#1}}
\newcommand*{\Ctx}[2]{\ensuremath{#1,\, #2}}
\newcommand*{\All}[2]{\ensuremath{∀ #1.\; #2}}
\newcommand*{\AllE}[3]{\ensuremath{∀ #1 < #2.\; #3}}

\newcommand*{\ap}{\ensuremath{\;}}
\newcommand*{\fun}[2]{\ensuremath{\mathop{λ} #1.\; #2}}

\newcommand*{\lam}[2]{\ensuremath{\mathop{λ} #1.\; #2}}
\newcommand*{\app}[2]{\ensuremath{#1 \ap #2}}
\newcommand*{\slam}[2]{\ensuremath{\mathop{λ} #1.\; #2}}
\newcommand*{\sapp}[2]{\ensuremath{#1 \ap #2}}
\newcommand*{\zero}[1]{\ensuremath{\mathrm{zero} \ap #1}}
\newcommand*{\suc}[3]{\ensuremath{\mathrm{suc} \ap #1 \ap #2 \ap #3}}
\newcommand*{\case}[5]{\ensuremath{\mathrm{caseNat}[#1] \ap #2 \ap #3 \ap #4
    \ap #5}}
\newcommand*{\cons}[3]{\ensuremath{\mathrm{cons} \ap #1 \ap #2 \ap #3}}
\newcommand*{\head}[2]{\ensuremath{\mathrm{head} \ap #1 \ap #2}}
\newcommand*{\tail}[3]{\ensuremath{\mathrm{tail} \ap #1 \ap #2 \ap #3}}
\newcommand*{\fix}[3]{\ensuremath{\mathrm{fix}[#1] \ap #2 \ap #3}}

\newcommand*{\V}[1]{\ensuremath{\mathit{#1}}}

\DeclareMathOperator{\Type}{Type}
\DeclareMathOperator{\refl}{refl}
\DeclareMathOperator{\id}{id}
\DeclareMathOperator{\PRGraphs}{PRGraphs}
\DeclareMathOperator{\PRGraphFams}{PRGraphFams}
\DeclareMathOperator{\Ap}{Ap}
\DeclareMathOperator{\curry}{curry}
\DeclareMathOperator{\eval}{eval}
\DeclareMathOperator{\Size}{Size}
\DeclareMathOperator{\wk}{wk}
\DeclareMathOperator{\Weaken}{Weaken}
\DeclareMathOperator{\Lift}{Lift}
\DeclareMathOperator{\Wk}{Wk}
\DeclareMathOperator{\Id}{Id}
\DeclareMathOperator{\Skip}{Skip}
\DeclareMathOperator{\Fill}{Fill}
\DeclareMathOperator{\mssuc}{ssuc}
\DeclareMathOperator{\Param}{Param}
\DeclareMathOperator{\Cats}{Cats}
\DeclareMathOperator{\Sets}{Sets}
\DeclareMathOperator{\Sizes}{Sizes}

\newcommand*{\Op}[1]{#1^{\mathord{\mathrm{op}}}}