\chapter{Background}
\label{sec:background}


\section{Termination Checking}
\label{sec:background:termination}

Proof assistants based on dependently typed languages must, if they are to be
consistent, ensure that all recursive programs terminate. The usual solution to
this problem, used by Coq, Agda, Idris, Lean and others, is to augment the type
checker with a separate termination checker based on the principle of structural
recursion. That principle says roughly: a program terminates if the input to any
recursive call is a subterm of the original input. For example, consider the
usual definition of addition for natural numbers:
\begin{code}
  plus : ℕ → ℕ → ℕ
  plus zero    m = m
  plus (suc n) m = suc (plus n m)
\end{code}

In the recursive call in the second equation, \icode{n} is a subterm of the
original input, \icode{suc~n}. Since this is the only recursive call, the
definition as a whole is structurally recursive and is accepted by Agda's
termination checker. This is justified because the natural numbers are
inductively defined, so any closed term of type \icode{ℕ} consists of finitely
many applications of the \icode{suc} constructor. If we \enquote*{peel off} one
constructor for every recursive call, we must eventually reach the \icode{zero}
case and the recursion ends.

Structural recursion as a basis for termination checking has many advantages. It
is conceptually simple, relatively easy to implement and surprisingly versatile:
many interesting definitions are naturally structurally recursive. However,
structural termination checkers are also inadequate in some important ways.

First, structural recursion is based on the syntactic notion of a subterm. This
makes termination checking non-compositional: we cannot, in general, replace a
term with another term of the same type in a recursive definition and still
expect the definition to typecheck. For a contrived example, take our previous
definition of \icode{plus} and replace the \icode{n} by \icode{id~n} in the
recursive call. The identity function \icode{id} just returns its argument, so
the two terms are obviously equivalent, but a naive termination checker would
not accept the modified definition -- after all, \icode{id~n} is not a syntactic
subterm of \icode{suc~n}. (Agda's termination checker is smart enough to
evaluate \icode{id~n} to \icode{n} and therefore accepts the modified
definition.)

Unfortunately, this non-compositionality bites not only in such contrived
situations. An often-cited example is the mapping function on rose trees (for
which we also need the standard list type and its mapping function):
\begin{code}
  data List (A : Set) : Set where
    []  : List A
    _∷_ : A → List A → List A

  mapList : ∀ {A B} → (A → B) → List A → List B
  mapList f []       = []
  mapList f (x ∷ xs) = f x ∷ mapList f xs

  data Tree (A : Set) : Set where
    leaf : A → Tree A
    node : List (Tree A) → Tree A

  mapTree : ∀ {A B} → (A → B) → Tree A → Tree B
  mapTree f (leaf x)  = leaf (f x)
  mapTree f (node xs) = node (mapList (mapTree f) xs)
\end{code}
Agda's termination checker does not accept \icode{mapTree}, and with good
reason: the recursive call in the second equation does not involve a subterm of
any input. We can argue that when we unfold \icode{mapList}, it becomes apparent
that \icode{mapTree} is always applied to an element of the list \icode{xs} and
these elements are obviously subterms of \icode{xs}. But Agda's syntactic check
is not smart enough to realise this. (Coq's termination checker employs a
heuristic that can deal with \icode{mapTree} by essentially inlining a
specialised version of \icode{mapList}.)

A second problem of termination checkers based on structural recursion is that
it is tempting to extend them with various heuristics to deal with more
complicated variants of structural recursion such as mutually recursive
definitions, lexicographic termination measures and nested inductive datatypes
like our rose trees. As a result, the termination checkers of Coq and Agda have
become quite complex, and with complexity come bugs: both Coq and Agda have
shipped versions with errors in their termination checkers that made the systems
inconsistent \cite{coqbug2013,agdabug2013}.

The final problem with structural recursion is that it does not easily
accommodate coinductive datatypes and corecursive definitions. Corecursive
functions produce values of coinductive types, which can be infinite. For
example, we can coinductively define the type of streams whose values are
infinite lists. A function which produces a stream obviously cannot terminate.
Instead, we must demand that the function is \emph{productive}, meaning that it
generates any finite prefix of its output in finite time. In other words, any
finite observation of an infinite stream must terminate. There are syntactic
methods, analogous to structural recursion, to ensure productivity, but they are
very inflexible. Sized types, on the other hand, when combined with copatterns
\cite{abel2016}, yield a practical productivity checker mostly for free.

To address these problems of structural termination checkers, we look to sized
types as a type-based termination checking mechanism.


\section{Sized Types}
\label{sec:background:sized}

Sized types are a termination checking methodology that does not rely on a
syntactic analysis. Instead, terms of inductive and coinductive types are
annotated at the type level with a size. For terms of inductive types, this size
may be thought of as an upper bound on the height of the term, viewed as a
constructor tree. For coinductive types, the size denotes the maximum depth to
which the term may be inspected. With this setup, checking a recursive
definition becomes easy: a recursive call is justified if the size of its
argument decreases. A corecursive call is justified if it increases the maximum
observation depth of its result.

Various type systems based on these principles have been proposed (see
\secref{conclusion:related} for references). This thesis investigates a calculus
whose sized types closely resemble Agda's, so the remainder of this section
gives a brief overview of Agda's system. More details and examples can be found
in Agda's manual \cite{agdamanual} and in \cite{abel2016}.

A size is a term of type \icode{Size<~n}, where \icode{n} is also a size.
Primitive sizes are variables, the successor of a size \icode{↑~n} and
\enquote*{infinity}, \icode{∞}. The size \icode{∞} plays a special role: it
designates \enquote*{fully defined} types, i.e.\ those whose values could have any
size.

For a size \icode{m} to have type \icode{Size<~n} it must be less than
\icode{n} according to an order \icode{<}. This order is mostly straightforward;
for example, we have \icode{n~<~↑~n} for all \icode{n}. However, we also have
\icode{n~<~∞} for all \icode{n} and in particular \icode{∞~<~∞}. As we will see,
this rule creates significant problems.

A sized inductive type is simply an inductive type with a parameter of type
\icode{Size} (which is the same as \icode{Size<~∞}). Continuing our rose tree
example, we define sized lists and a mapping function:
\begin{code}
  data Listₛ (A : Set) (n : Size) : Set where
    []   : Listₛ A n
    cons : (m : Size< n) → A → Listₛ A m → Listₛ A n

  mapListₛ : ∀ {A B} n → (A → B) → Listₛ A n → Listₛ B n
  mapListₛ n f []            = []
  mapListₛ n f (cons m x xs) = cons m (f x) (mapListₛ m f xs)
\end{code}
The typing of the \icode{cons} constructor reflects the intuition that the
height of \icode{cons~m~x~xs}, \icode{n}, is strictly greater than the height of
\icode{xs}, \icode{m}.\footnote{Agda also supports a different style of using
  sized types where \icode{cons} takes a \icode{Listₛ~A~n} as input and returns
  a \icode{Listₛ~A~(↑~n)} (making \icode{n} an index rather than a parameter of
  \icode{Listₛ}). This works just as well for the most part, but there are
  technical reasons to prefer our \enquote*{quantifier style}.} We can then
exploit this in \icode{mapListₛ}: in the recursive call, \icode{mapListₛ} is
applied to \icode{m}, which we know from the type of \icode{cons} is less than
\icode{n}. Thus, each recursive call decreases in size and the recursion is
justified. The definition also demonstrates that the order on sizes induces a
subtyping relation: the right-hand side of the second equation has type
\icode{Listₛ~A~m}, which is a subtype of \icode{Listₛ~A~n} for \icode{m~<~n}.

So far, we could have just as well used a structural termination checker. The
benefits of sized types become obvious when we turn to rose trees:
\begin{code}
  data Treeₛ (A : Set) (n : Size) : Set where
    leaf : A → Treeₛ A n
    node : (m : Size< n) → Listₛ (Treeₛ A m) ∞ → Treeₛ A n

  mapTreeₛ : ∀ {A B} n → (A → B) → Treeₛ A n → Treeₛ B n
  mapTreeₛ n f (leaf x)    = leaf (f x)
  mapTreeₛ n f (node m xs) = node m (mapListₛ ∞ (mapTreeₛ m f) xs)
\end{code}
The definition of \icode{Treeₛ} demonstrates the utility of \icode{∞}: we can
say that a \icode{node} should have a list of children without caring about the
size of that list. But more importantly, the termination checker now accepts
\icode{mapTreeₛ}: the recursive call is now at size \icode{m~<~n} and thus
justified. In effect, we have encoded in the type of \icode{node} our intuition
that the height of \icode{node~xs} is strictly greater than the height of any of
the elements of \icode{xs}.

Agda's sized types thus deliver on the compositionality promise: all information
the termination checker needs is encoded at the type level, so we can freely
abstract over terms. Sized types are also mostly straightforward to implement,
being just an extension of the type system.\footnote{I say \enquote{mostly}
  because Agda's current implementation includes some subtle checks to prevent
  inconsistent size assumptions, which is necessary to preserve decidability of
  type checking. Agda also has a sophisticated size inference engine (so we
  could have left all sizes in our example code implicit), but this engine is
  not part of the trusted computing base.}

Unfortunately, such convenience currently comes at the ultimate price: Agda's
implementation of sized types is, and has been for some time, inconsistent. The
culprit seems to be the highly dubious rule \icode{n~<~∞}, in particular
\icode{∞~<~∞}. This rule makes the \icode{<} relation obviously non-well-founded
(meaning there is an infinite descending chain \icode{∞~<~∞~<~\dots}) but Agda
assumes that \icode{<} is well-founded. This assumption can be exploited in
different ways \cite{agdabug2015,agdabug2016,agdabug2017,agdabug2018} to sneak
non-terminating programs past the termination checker.

It is currently unclear how to satisfactorily resolve this issue with Agda's
design. In this thesis, I adopt the obvious solution: changing the \icode{<}
relation so that \icode{∞~≮~∞}. Doing the same in Agda would, however, lead to
issues with the constructors and fields of sized data types. For example, we
would like to use the \icode{cons} constructor for sized lists at type
\begin{code}
  A → Listₛ A ∞ → Listₛ A ∞
\end{code}
but this is impossible with \icode{∞~≮~∞}. As a workaround, we can define a
consing operation for lists at \icode{∞} that is extensionally equal to
\icode{cons}, but this requires a pattern match on the input list -- an
inefficiency that should not be necessary. How best to \enquote*{rescue} Agda's
sized types thus remains an open question for now.

Another possible criticism of Agda's sized types concerns expressivity. Agda's
size arithmetic is restricted to the successor; we cannot add or multiply sizes.
This means that we cannot give a precise type to, for example, the list
appending function, whose output size should be the sum of its input sizes. This
lack of expressivity, however, is a conscious design decision. In return, we get
a size inference algorithm that can infer almost all sizes in a typical program.
This significantly lowers the cost of adopting sized types.
