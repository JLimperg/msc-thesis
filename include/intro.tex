\chapter{Introduction}

Dependent type theories are now well-established as a technology for certifying
software and formalising mathematics. Yet they remain rough around some edges,
one of which is termination checking. If a type theory is to be consistent as a
logic, it must ensure that all recursive programs terminate (and that all
corecursive programs are productive). In mainstream dependently typed languages
such as Coq, Agda, Idris and Lean, this termination check is implemented by
heuristics based on a simple syntactic criterion for termination, the principle
of structural recursion.

Despite its simplicity, structural recursion is surprisingly powerful in
practice. Still, it suffers from some flaws that are directly related to its
syntactic nature. It makes type checking non-compositional: We cannot always
replace a term by another term of the same type because some terms have special
meaning to the termination checker. It does not play well with more complex data
types such as the nested inductive type of rose trees. It does not easily
accommodate corecursive definitions, which require a separate productivity
check. And when structural checkers are extended to handle more complex
scenarios, such as mutual and nested recursion, they tend to become complex and
therefore hard to implement correctly. For more details on structural recursion
and its shortcomings, see \secref{background:termination}.

To address the problems with structural recursion, several other termination
checking regimes have been proposed. One of these is \emph{sized types}, an
umbrella term for a family of broadly similar type systems that use type
annotations to ensure termination. In these systems, inductive types are
annotated with a size: $\Nat{n}$, for example, is the type of natural numbers
with size $n$. For the moment, we can view $n$ as a natural number and say that
$\Nat{n}$ contains only natural numbers $m ≤ n$. This interpretation suggests a
simple termination criterion: If, in a recursive definition, we receive an input
in $\Nat{n}$ and only recurse on terms in $\Nat{m}$ for some $m < n$, then the
recursion must stop when it reaches the size zero (at the latest).

Since this termination check is based entirely on type-level information, it
avoids the non-compositionality of syntactic approaches. It also handles nested
inductive data types well and, when combined with copatterns \cite{abel2016},
can be used to check productivity of corecursive definitions in a similarly
natural way. The dependently typed language Agda features an implementation of
sized types that demonstrates these advantages (though see
\secref{background:sized} for some problems of this implementation, as well as
more details on Agda's sized types).

Motivated by the desirable properties of sized types, this thesis investigates a
custom lambda calculus with sized types called λST, defined in \chapref{source}.
The calculus extends the simply-typed lambda calculus with sized types that
closely resemble Agda's. As such, it provides a setting for
\enquote*{experiments} with Agda's sized types: Investigations of the properties
of sized types without the surrounding complexity of a full dependently-typed
language.

Of these properties, two are of special interest. The first is normalisation:
The type system should ensure that all programs do actually terminate. This is
the raison d'être of sized types and thus the focus of existing work . The
second property we are interested in is size irrelevance: Sizes should only be
used during type checking to ensure termination/productivity and should not
affect the runtime behaviour of programs. This means that sizes can be erased at
runtime, which is good for performance.

In \chapref{model}, I focus on size irrelevance as the less-studied of the two
properties. I demonstrate that λST is size-irrelevant using a reflexive graph
model. This is a technique which is usually used to establish parametricity
results, which fits the problem well: Size irrelevance just means that any term
which depends on a size is parametric in that size. The model is fully
formalised in Agda (without sized types); \chapref{formalisation} discusses the
formalisation and some of its technical challenges.

Before I settled on the reflexive graph approach, I also investigated two
category-theoretical modelling approaches which looked promising, but turned out
to be inadequate. \chapref{negative} briefly discusses these models and their
problems.
