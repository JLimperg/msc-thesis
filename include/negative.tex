\chapter{Negative Results}
\label{sec:negative}

Before developing the reflexive graph model of λST, I explored two other
modelling approaches which turned out, for different reasons, to be unsuited to
the task at hand. They are both based on the idea of interpreting size contexts
as preorder categories, types as functors from a size context category into the
category of sets and terms as natural transformation between such functors. A
model like this would yield a sort of size irrelevance in the form of the
naturality of term interpretations: if the interpretation of a term in size
context $Δ$ is natural in $⟦Δ⟧$, then $⟦Δ⟧$ must be essentially irrelevant. In
the next two sections, I briefly describe these potential models and why they do
not, in fact, model λST. This is in somewhat surprising contrast to the work of
Veltri and van der Weide \cite{veltri2019}, who model guarded recursion --
another form of type-based termination checking that is quite similar to sized
types -- using essentially the same approach that fails for λST.


\section{Covariant Presheaf Model}
\label{sec:negative:covariant}

Our first potential model starts with the observation that size contexts have a
natural interpretation as categories. Recall our previous interpretation of size
contexts as types:
\begin{Align*}
  ⟦()⟧ &=& ⊤ \\
  ⟦\ctx{Δ}{n}⟧ &=& \Sigma_{δ ∈ ⟦Δ⟧} \Size_{<⟦n⟧(δ)}.
\end{Align*}
We can extend this interpretation from types to categories: $⟦()⟧$ is the
terminal category (whose object type is $⊤$) and $⟦\ctx{Δ}{n}⟧$ is the
Grothendieck category of the functor $F ∶ ⟦Δ⟧ → \Cats$ with $F(δ) ≔
\Sizes_{<⟦n⟧(δ)}$. ($\Cats$ is the category of small categories.) Sizes are
modelled as functors $⟦n⟧ ∶ ⟦Δ⟧ → \Sizes$ where $\Sizes$ is the category of
sizes ordered by the preorder $≤$; $\Sizes_{<n}$ is the subcategory of $\Sizes$
which contains only sizes less than $n$. The objects of $⟦Δ⟧$ are then
telescopes of sizes, as before, and an arrow $f ∶ δ → δ′$ in $⟦Δ⟧$ is a proof
that the sizes in $δ$ are pointwise less than or equal to the sizes in $δ′$. The
judgment $Δ ⊢ n < m$ is modelled by a proof that $⟦n⟧(δ) < ⟦m⟧(δ)$ for arbitrary
$δ ∶ ⟦Δ⟧$.

Continuing the categorical interpretation, we model types in a size context $Δ$
as functors from $⟦Δ⟧$ to $\Sets$, the category of sets (or rather, since we work
in type theory, types). These are \enquote*{covariant presheaves}, meaning
presheaves over the opposite category of $⟦Δ⟧$, $\Op{⟦Δ⟧}$.

One problem with this approach becomes apparent already at this stage: the type
$\Stream{n}$ of sized streams has no natural interpretation as a covariant
functor from $⟦Δ⟧$ to $\Sets$. We would like to define, as before,
\begin{displaymath}
  ⟦\Stream{n}⟧(δ) ≔ ℕ_{<⟦n⟧(δ)} → ℕ
\end{displaymath}
but this is no functor: for $δ ≤ δ′$ we have $⟦n⟧(δ) ≤ ⟦n⟧(δ′)$ by functoriality
of $⟦n⟧$, but there is no appropriate function from $ℕ_{<⟦n⟧(δ)} → ℕ$ to
$ℕ_{<⟦n⟧(δ′)} → ℕ$. This is not surprising -- after all, ⟦\Stream{n}⟧(δ) is
naturally contravariant in $δ$.

Still, let us ignore this problem and move on to discover more fundamental
issues. The other types of λST have more or less natural interpretations:
\begin{Align*}
  ⟦\Nat{n}⟧(δ) &≔& ℕ_{≤⟦n⟧(δ)} \\
  ⟦T → U⟧(δ) &≔& \All{δ′ ≥ δ}{⟦T⟧(δ′) → ⟦U⟧(δ′)} \\
  ⟦\All{n}{T}⟧(δ) &≔& \All{δ′ ≥ δ}{\All{m < ⟦n⟧(δ′)}{⟦T⟧(δ′, m)}}.
\end{Align*}
Sized natural numbers $\Nat{n}$ are interpreted as before. The function space $T
→ U$ is modelled by the exponential of presheaves, which ensures that we can
also model abstractions and applications. The exponential is defined using a
monotonisation \enquote*{trick}: the function space $⟦T⟧(δ) → ⟦U⟧(δ)$ would not
be functorial due to the negative occurrence of $δ$, but we can force
functoriality by quantifying over $δ′ ≥ δ$. The same approach also allows us to
interpret size quantification, where again the natural interpretation without
monotonisation would not be functorial.

Unfortunately, while this monotonisation trick works for the exponential, it
does not yield an appropriate model of size quantification. This is not very
surprising: in the above interpretation of $\All{n}{T}$, the size that we
introduce in the model, $m$, is not, in general, smaller than $⟦n⟧(δ)$. The
bound $m < ⟦n⟧(δ′)$ does not actually restrict the domain of $m$ since $δ′ ≥ δ$
and thus $⟦n⟧(δ′) ≥ ⟦n⟧(δ)$. The reader is invited to check that this prevents
us from interpreting $\mathrm{fix}$.


\section{Contravariant Presheaf Model}
\label{sec:negative:contravariant}

Seeing that the major problem with the previous model was the need to
force-monotonise the interpretation of size quantification, we now consider a
model based on \emph{contravariant} presheaves (i.e.\ just presheaves). Size
quantification is naturally contravariant, so no trickery is necessary to
interpret it. Indeed, this approach leads to a more satisfactory interpretation
of types (sizes, size contexts etc.\ are modelled as before):
\begin{Align*}
  ⟦\Stream{n}⟧(δ) &≔& ℕ_{≤⟦n⟧(δ)} → ℕ \\
  ⟦T → U⟧(δ) &≔& \All{δ′ ≤ δ}{⟦T⟧(δ′) → ⟦U⟧(δ′)} \\
  ⟦\All{n}{T}⟧(δ) &≔& \All{m < ⟦n⟧(δ)}{⟦T⟧(δ, m)}.
\end{Align*}

With the switch to contravariant functors, we lose the ability to model sized
natural numbers, but gain the ability to model sized streams. As before, the
function space is interpreted as an exponential of presheaves. Size
quantification does not require monotonisation any more, and indeed this
approach appears, on the surface, to support natural interpretations of all
terms including $\mathrm{fix}$.\footnote{The above interpretation of size
  quantification is not quite accurate: to successfully model λST, we would need
  to restrict it to allow only size-parametric functions, as in the reflexive
  graph model. However, this is unimportant for the rest of the discussion.}

Alas, the contravariant presheaf approach still fails, this time due to a more
subtle problem: the substitution lemma for size substitution in types does not
hold. This means that in general we have
\begin{displaymath}
  ⟦\sub{T}{σ}⟧ ≠ \sub{⟦T⟧}{⟦σ⟧}.
\end{displaymath}

To see why, we first consider the interpretation of substitutions. Recall that
in the reflexive graph model, a well-typed substitution $σ ∶ Δ ⇒ Ω$ was
interpreted as a function between the types $⟦Δ⟧$ and $⟦Ω⟧$. Here, we upgrade
this interpretation to a functor between the categories $⟦Δ⟧$ and $⟦Ω⟧$, but the
underlying function remains the same. Semantic substitution is composition:
given a type $T$ in size context $Ω$, whose interpretation is a functor from
$⟦Ω⟧$ to $\Sizes$, we define
\begin{displaymath}
  \sub{⟦T⟧}{⟦σ⟧} ≔ ⟦T⟧ ∘ ⟦σ⟧.
\end{displaymath}

Now consider the substitution which assigns to the zeroth variable (in an
otherwise empty context) the size $1$:
\begin{displaymath}
  σ ≔ \Sing(\ssuc{0}) ∶ () → \ctx{()}{∞}.
\end{displaymath}
Its interpretation is $⟦σ⟧(()) = ((), 1)$, where $()$ is the sole inhabitant of
the unit type. Further, let $T$ be the following type (in the context
$\ctx{()}{∞}$) of functions from streams to an arbitrary closed type $U$:
\begin{displaymath}
  T ≔ \Stream{v_0} → U.
\end{displaymath}
Then we have
\begin{align*}
  ⟦\sub{T}{σ}⟧(())
    &= ⟦\Stream{1} → U⟧(()) \\
    &= \All{() ≤ ()}{⟦\Stream{1}⟧(()) → ⟦U⟧(())} \\
    &= \All{() ≤ ()}{(\All{k ≤ 1}{ℕ}) → ⟦U⟧(())} \\
    &≅ (\All{k ≤ 1}{ℕ}) → ⟦U⟧(())
  \\
  \sub{⟦T⟧}{⟦σ⟧}(())
    &= ⟦T⟧(⟦σ⟧(())) \\
    &= ⟦\Stream{v_0} → U⟧((), 1) \\
    &= \All{m ≤ 1}{⟦\Stream{v_0}⟧((), m) → ⟦U⟧((), m)} \\
    &= \All{m ≤ 1}{(\All{k ≤ m}{ℕ}) → ⟦U⟧(())}.
\end{align*}
These two types are not isomorphic, so substitutions do not have the desired
semantics.
