% CREATED BY DAVID FRISK, 2016
\chapter{Conclusion}
\label{sec:conclusion}

\section{Related Work}
\label{sec:conclusion:related}

\paragraph{Sized Types} Sized types go back to Hughes, Pareto and Sabry
\cite{hughes1996}, who coined the term and introduced the first calculus
featuring sized types. Since then, a variety of calculi have been investigated,
for example by Amadio and Coupet-Grimal \cite{amadio1998}; Barthe, Frade,
Giménez, Pinto and Uustalu \cite{barthe2004}; Blanqui \cite{blanqui2004}; and
Sacchini \cite{sacchiniphd, sacchini2013}. These works all use slightly
different notions of sizes on top of different base calculi (simply typed or
dependently typed, with inductive or coinductive types, etc.), but none of them
address the question of size irrelevance. Agda's particular notion of sized
types, which λST emulates, is due to Abel and Pientka \cite{abel2016}.

\paragraph{Reflexive Graph Models} Reynolds \cite{reynolds1983} first
introduced the technique of interpreting a type as a set together with a
relation on that set. This is usually called a relationally parametric model.
Atkey, Ghani and Johann \cite{atkey2014} apply this technique to a dependent
type theory, showing in particular how families of reflexive graphs can be used
to interpret type dependencies. Vezzosi \cite{vezzosi2015} gives a reflexive
graph model of a type theory with guarded recursion, another type-based
termination checking mechanism. Nuyts, Vezzosi and Devriese \cite{nuyts2017}
present a dependent type theory with parametric quantifiers for which they also
give a reflexive graph model. Their calculus allows users to reason internally
about parametricity, which enables an encoding of sized types as types
parametrically indexed by natural numbers (though they do not discuss the
special size ∞).


\section{Future Work}
\label{sec:conclusion:future}

This thesis is but a first step towards a realistic account of Agda's sized
types. In this section, I discuss some possible directions for future work
(besides the obvious extensions of λST with dependent types, data types and so
forth).


\paragraph{Normalisation}

This thesis is only concerned with one of the two interesting properties of λST,
size irrelevance. The other is normalisation, which is arguably more important
but also much better covered in the literature. In the case of λST, one would
have to be careful with the definition of normalisation since its streams are
infinite by design.


\paragraph{Syntactic Size Irrelevance}

As mentioned in \secref{model:conclusion}, this thesis does not, in fact, prove
any interesting properties of λST -- only of its model. To connect the model
with a syntactic notion of size irrelevance, one might proceed as follows:
\begin{enumerate}
  \item Give an operational semantics for λST.
  \item Prove that if two terms are β-equal, their models are equal. (This is
    almost trivial informally but difficult to formalise.)
  \item Give a notion of syntactic equality up to sizes.
  \item Prove a lemma that allows us to conclude syntactic equality up to sizes
    from semantic equality.
\end{enumerate}

Steps 1--3 are not hard, at least conceptually. For step 4, it is not clear to
me exactly how to connect the model back to the syntax. One candidate is the
following conjecture: for $u ∶ \Nat{m}$ and $u′ ∶ \Nat{m′}$ (in an empty
context), if $u$ and $u′$ are normal and $⟦u⟧ = ⟦u′⟧$, then $u$ and $u′$ are
syntactically equal up to sizes. This, together with the conjecture that
β-equality implies equality in the model, would give us a weak form of size
irrelevance restricted to base types and normal forms: if we have $t ∶
\All{n}{\Nat{v_0}}$ and $\sapp{t}{m}$ normalises to $u$ and $\sapp{t}{m′}$
normalises to $u′$, then $u$ and $u′$ are equal up to sizes.


\paragraph{Infinitely Branching Data Types}

Natural numbers and streams, viewed as constructor trees, are finitely
branching: each constructor has a finite number of subterms (one or zero in the
case of natural numbers; one in the case of streams). However, Agda and other
dependently typed languages also feature infinitely branching types such as this
one:
\begin{code}
  data ℕTree (A : Set) : Set where
    leaf : A → ℕTree A
    node : (ℕ → ℕTree A) → ℕTree A
\end{code}
\icode{ℕTree~A} is a type of trees similar to rose trees except that each
\icode{node} has countably infinite children.

Such infinitely branching types complicate the interpretation of sizes. Recall
that we interpreted sizes as the height of a constructor tree. For finitely
branching types, this height is always a (computable) natural number. For
infinitely branching types such as \icode{ℕTree}, however, it is a noncomputable
ordinal: the height of a \icode{node} is one plus the maximum height of its
infinite collection of child trees. This means that in the model, $∞$ could no
longer be interpreted as $ω$; instead it would have to be an ordinal that is
greater than the height of any possible value of an infinitely branching data
type.

Ordinals present some technical challenges since they are not trivially encoded
in type theory. As part of the work for this thesis, I investigated a particular
subclass of ordinals, the plump ordinals \cite{taylor1996} as presented by
Shulman \cite{shulman2014}, as candidates for a model of sizes. A formalisation
of these ordinals can be found in \texttt{Ordinal.\allowbreak Shulman}. However,
the investigation was inconclusive: while the formalisation shows that the plump
ordinals have most of the properties we expect of a model of sizes, they do not
admit the seemingly straightforward lemma that for ordinals $n < m$ and $n′ <
m$, their supremum $n ⊔ n′$ is also less than $m$. More precisely, this lemma is
equivalent to classical logic -- but then again, it is not clear that we need it
to interpret a hypothetical extension of λST with infinitely branching types.


\section{Conclusion}
\label{sec:conclusion:conclusion}

I have presented a calculus, λST, which approximates Agda's sized types in a
simply-typed setting. This calculus has a reflexive graph model which gives a
size-irrelevant denotational semantics. Both λST and the model are fully
formalised.
