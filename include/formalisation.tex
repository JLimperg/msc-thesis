\chapter{Formalisation}
\label{sec:formalisation}

All results from the previous sections are formalised in the dependently typed
proof assistant Agda. The formalisation is freely available online
\cite{limperg2019code} and closely follows the informal discussion in this
thesis, so in this chapter I only discuss the major design decisions that shape
it.


\section{Metatheory}
\label{sec:formalisation:metatheory}

λST is formalised in Agda \cite{norellphd}, a proof assistant based loosely on
Luo's UTT \cite{luo1992}. I use version 2.6.0.1 with axiom K disabled via the
\icode{--without-K} flag. This configuration disables Agda's sized types, so we
do not reason circularly.

I extend the metatheory in one major way, by postulating the axiom of univalence
\cite{hottbook}. Type theory with univalence has a model in simplicial sets
\cite{kapulkin2012}, so this should be safe (as long as axiom K, which
contradicts univalence, is disabled). Univalence allows us to avoid some of the
boilerplate that is often needed when interpreting set-theoretical mathematics
in type theory. In particular, with univalence, propositional equality is the
correct notion of equality for all the constructions we work with, such as
streams and reflexive graphs. Without univalence, we would have to define custom
equivalence relations for these structures, which would in turn force us to
parameterise the definitions of reflexive graphs and families of reflexive
graphs by an equivalence relation. This is doable -- and I conjecture that the
formalisation could be adapted more or less mechanically to a type theory
without univalence -- but the resulting complications would obscure the
constructions.

As an alternative to postulated univalence, we could also work in Cubical Agda
\cite{vezzosi2019}, an extension of Agda that admits univalence as a theorem.
Compared with postulated univalence, Cubical Agda has the advantage that more
equations hold definitionally rather than just propositionally. However, in our
particular case, we would not use these additional definitional equalities much.
On the other hand, switching to Cubical Agda would complicate the interoperation
with Agda's standard library, which does not use the cubical primitives
(specifically the cubical identity type).

To benefit from univalence, I reimplement some basic definitions and results
from Homotopy Type Theory such as H-Levels, the theory of equivalences and the
derivation of functional extensionality from univalence. For reference, see the
Homotopy Type Theory book \cite{hottbook}. Much of my code is inspired by Martín
Escardó's implementation of Homotopy Type Theory in Agda \cite{escardo2019}.


\section{Size Substitutions}
\label{sec:formalisation:sub}

The one instance where the formalisation departs somewhat from the account in
this thesis concerns size substitutions. If we were to stick exactly to the
text, we would face the following problem: when we apply the interpretation of
the weakening substitution $\Wk$ to a size valuation $(δ, n)$, we would get
$⟦\Wk⟧(δ, n) = δ$ as a \emph{propositional} equality, but not as a
\emph{definitional} equality (and similar for the other substitutions). This
matters because $⟦\Wk⟧(δ, n)$ sometimes appears in types, in which case we would
have to use the transport operator \icode{subst} to define terms of these types.
This, in turn, makes it hard to prove facts about these terms.

To avoid this issue, we use a universe of size substitutions, formalised in
\module{Source.\allowbreak Size.\allowbreak Substitution.\allowbreak
  Universe}{Source.Size.Substitution.Universe}. We define a data type
\icode{Sub} that has a constructor for each of the particular substitutions we
need (e.g.\ $\Wk$, $\Sing$). We then define an interpretation function which maps
each value of this datatype to the corresponding \enquote*{canonical}
substitution from \secref{source:sub}. The application of a substitution in
\icode{Sub} to a size/type/context/term is then defined simply as the
application of its canonical interpretation.

This setup gives us the best of both worlds: all lemmas about canonical
substitutions are easily transferred to the universe construction, but when we
model substitutions in \icode{Sub} as PRGraph morphisms, we can \emph{define}
$⟦\Wk⟧(δ, n) ≔ n$. This is exactly the definitional equality we wanted.


\section{Interpretation of the Fixpoint}
\label{sec:formalisation:fixpoint}

In \secref{model:terms:fixpoint}, I mentioned that the interpretation of the
term $\mathrm{fix}$ is not quite as simple as I made it appear. Recall that the
typing rule of $\mathrm{fix}$ lets us assume a morphism $⟦t⟧$ with components
\begin{displaymath}
  ⟦t⟧_δ ∶ ⟦Γ⟧(δ) → (n ∶ \Size_{<ω + 1}) → ((m ∶ \Size_{<n}) → ⟦T⟧(δ, m)) → ⟦T⟧(δ, n).
\end{displaymath}
We then defined the following fixpoint of $⟦t⟧$:
\begin{gather*}
  f ∶ ⟦Γ⟧(δ) → (n ∶ \Size_{<ω + 1}) → ⟦T⟧(δ, n) \\
  f(γ)(n) ≔ ⟦t⟧(γ)(n)(\fun{m}{f(γ)(m)}).
\end{gather*}

This definition is easily replicated in Agda using well-founded recursion
\cite{paulson1986,nordstroem1988}, a technique that is commonly used to define
non-structurally recursive functions. We will use a variation of this technique,
so let us recall the basics first.

Well-founded recursion encodes the idea that given a well-founded relation $<$,
a recursive definition which decreases this relation with every recursive call
must terminate -- for if it did not, we would have an infinite descending chain
$n₀ > n₁ > \dots$ and \enquote{well-founded} means precisely that no such chain
exists. Our definition of $f$ above follows this scheme: the recursive call is
applied to $m < n$ and since $<$ is well-founded, the recursion must eventually
stop.

To represent this idea in Agda, we first define what it means for a relation to
be well-founded. To that end, we introduce the auxiliary concept of
\Def{accessibility} in the form of the \icode{Acc} data type. This type can be
found in the module \module{Induction.\allowbreak
  WellFounded}{Induction.WellFounded} from Agda's standard library; the
remainder of this section is located either in the same module or in my
\module{Util.\allowbreak Induction.\allowbreak
  WellFounded}{Util.Induction.WellFounded}.
\begin{code}
data Acc (_<_ : A → A → Set) (x : A) : Set where
  acc : (∀ y → y < x → Acc _<_ y) → Acc _<_ x
\end{code}
This definition says that a term $x ∶ A$ is accessible with respect to the
relation $<$ if every $y < x$ is also accessible. \icode{Acc} is inductively
defined, which means that a proof of accessibility can \enquote*{stack} only
finitely many \icode{acc} constructors. Therefore, when a term is accessible,
any descending chain starting from that term must eventually end. This justifies
the following definition of well-foundedness: a relation $<$ on some type $A$ is
well-founded if every element of $A$ is accessible with respect to $<$.
\begin{code}
WellFounded : (A → A → Set) → Set
WellFounded _<_ = ∀ x → Acc _<_ x
\end{code}

We can then encode well-founded recursion. First, we need the following helper
function:
\begin{code}
wfInd-acc : (P : A → Set)
  → (∀ x → (∀ y → y < x → P y) → P x)
  → ∀ x → Acc _<_ x → P x
wfInd-acc P f x (acc rs) = f x (λ y y<x → wfInd-acc P f y (rs y y<x))
\end{code}
The definition of \icode{wfInd-acc} passes Agda's termination checker because it
recurses on the argument \icode{Acc~\_<\_~x}: in the recursive call, that argument
is \icode{rs~y~y<x}, which counts as a syntactic subterm of \icode{acc~rs}.
Bona fide well-founded recursion is a direct consequence of \icode{wfInd-acc}:
\begin{code}
wfInd : WellFounded _<_
  → (P : A → Set γ)
  → (∀ x → (∀ y → y < x → P y) → P x)
  → ∀ x → P x
wfInd <-wf P f x = wfInd-acc P f x (<-wf x)
\end{code}
The reader is invited to check that our earlier definition of the fixpoint $f$
is an instance of this recursion scheme.

However, that earlier definition is built on a white lie: The type of $⟦t⟧$ is
really more complex than advertised. Specifically, its argument of type $(m ∶
\Size_{<n}) → ⟦T⟧(δ, m)$ must be size-parametric. In the definition of $f$, that
argument is the function $\fun{m}{f(γ)(m)}$ -- so to define $f$, we must prove
that $f$ is size-parametric. Our well-founded recursion combinator \icode{wfInd}
is too weak for this situation.

We therefore need to construct a variation of \icode{wfInd} that allows us to
prove properties of a defined function while defining it. We call this variation
\icode{wfIndΣ}. Like \icode{wfInd}, it is an instance of a helper function,
\icode{wfIndΣ-acc}, which does most of the heavy lifting. That helper, and a
mutually defined proof about it, is shown in \figref{wfIndΣ}.

\begin{figure}
  \centering
\begin{code}
mutual
  wfIndΣ-acc : (P : A → Set) (Q : ∀ x y → P x → P y → Set)
    → (f : ∀ x
        → (g : ∀ y → y < x → P y)
        → (∀ y y<x z z<x → Q y z (g y y<x) (g z z<x))
        → P x)
    → (∀ x g g-resp y h h-resp
        → (∀ z z<x z′ z′<y → Q z z′ (g z z<x) (h z′ z′<y))
        → Q x y (f x g g-resp) (f y h h-resp))
    → ∀ x → Acc _<_ x → P x
  wfIndΣ-acc P Q f f-resp x (acc rs)
    = f x
      (λ y y<x → wfIndΣ-acc P Q f f-resp y (rs y y<x))
      (λ y y<x z z<x
          → wfIndΣ-acc-resp P Q f f-resp y (rs y y<x) z (rs z z<x))

  wfIndΣ-acc-resp : (P : A → Set) (Q : ∀ x y → P x → P y → Set)
    → (f : ∀ x
        → (g : ∀ y → y < x → P y)
        → (∀ y y<x z z<x → Q y z (g y y<x) (g z z<x))
        → P x)
    → (f-resp : ∀ x g g-resp y h h-resp
        → (∀ z z<x z′ z′<y → Q z z′ (g z z<x) (h z′ z′<y))
        → Q x y (f x g g-resp) (f y h h-resp))
    → ∀ x x-acc y y-acc
    → Q x y
        (wfIndΣ-acc P Q f f-resp x x-acc)
        (wfIndΣ-acc P Q f f-resp y y-acc)
  wfIndΣ-acc-resp P Q f f-resp x (acc rsx) y (acc rsy)
    = f-resp _ _ _ _ _ _ λ z z<x z′ z′<y
        → wfIndΣ-acc-resp P Q f f-resp z (rsx z z<x) z′ (rsy z′ z′<y)
\end{code}

  \caption{Implementation of wfIndΣ-acc and wfIndΣ-acc-resp}
  \label{fig:wfIndΣ}
\end{figure}

Compared to \icode{wfInd-acc}, the type of \icode{wfIndΣ-acc} has changed in the
following ways:
\begin{itemize}
  \item \icode{Q} is a heterogeneous relation on \icode{P}, the (dependent) type
    which we are constructing. In our use case, \icode{Q} will be instantiated
    with equality up to sizes.
  \item The function \icode{f}, which defines the computational behaviour of the
    fixpoint, still receives as an argument a function \icode{g} which
    represents a recursive call on a smaller argument. Additionally, \icode{f}
    now receives a proof that for any two smaller sizes \icode{y} and \icode{z},
    the recursive calls \icode{g~y} and \icode{g~z} are related by \icode{Q}. In
    our case, this means that \icode{g} is size-parametric. This is precisely
    the ingredient we were previously missing.
  \item The next argument to \icode{wfIndΣ-acc} is new. It demands a proof that
    \icode{f} preserves the relation \icode{Q}.
\end{itemize}

Mutually with \icode{wfIndΣ-acc}, we define a lemma about it,
\icode{wfIndΣ-acc-resp}. This just says that any two terms constructed by
\icode{wfIndΣ-acc} are related by \icode{Q} -- so in our case,
\icode{wfIndΣ-acc} constructs a size-parametric function. As with
\icode{wfInd-acc}, the implementations of \icode{wfIndΣ-acc} and
\icode{wfIndΣ-acc-resp} are straightforward.

It is now easy to define the fixpoint combinator \icode{wfIndΣ} based on
\icode{wfIndΣ-acc}. Defining $f$ in terms of it is also conceptually simple but
the details are a bit gnarly. For the full picture, see the module
\module{Model.\allowbreak Term}{Model.Term}.
