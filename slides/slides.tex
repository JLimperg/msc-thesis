\documentclass[xetex]{beamer}
\usepackage[british]{babel}
\usepackage{fontspec}
\usepackage{mathtools}
\usepackage{unicode-math}
\usepackage{minted}
\usepackage{csquotes}
\usepackage{mathpartir}

\setmainfont{DejaVu Serif}
\setsansfont{DejaVu Sans}
\setmathfont{XITS Math}
% Alternatively:
% \setmathfont{DejaVu Math TeX Gyre}
\setmonofont{DejaVu Sans Mono}

\beamertemplatenavigationsymbolsempty
\setbeamertemplate{itemize item}[square]
\setbeamertemplate{itemize subitem}[square]
\setbeamertemplate{itemize subsubitem}[square]

% Code environment
\newenvironment{code}
{\VerbatimEnvironment
  \begin{minted}{Agda}}
  {\end{minted}
  \par}

% crude replacement for align
\newenvironment{Align*}
{\begin{displaymath}\begin{array}{lcl}}
{\end{array}\end{displaymath}}

\newenvironment{AlignAnnot*}
{\begin{displaymath}\begin{array}{lcll}}
{\end{array}\end{displaymath}}

% Additional commands
\newcommand*{\secref}[1]{Sec.\ \ref{sec:#1}}
\newcommand*{\figref}[1]{Fig.\ \ref{fig:#1}}
\newcommand*{\chapref}[1]{Ch.\ \ref{sec:#1}}
\DeclarePairedDelimiter{\brackets}{[}{]}
\DeclarePairedDelimiter{\anglebrackets}{⟨}{⟩}

\newcommand*{\bnfdef}{\ensuremath{\Coloneqq}}
\newcommand*{\icode}[1]{\mbox{\texttt{#1}}}
\newcommand*{\Def}[1]{\emph{#1}}

\newcommand*{\ssuc}[1]{\ensuremath{\mathop{↑} #1}}
\newcommand*{\ctx}[2]{\ensuremath{#1,\, #2}}
\newcommand*{\ctxE}[3]{\ensuremath{#1,\, #2 < #3}}
\newcommand*{\ssub}[2]{\ensuremath{#1, #2}}
\newcommand*{\sub}[2]{\ensuremath{#1\brackets*{#2}}}
\newcommand*{\fcomp}{\ensuremath{\mathbin{\gg}}}

\newcommand*{\Nat}[1]{\ensuremath{\mathrm{Nat}~#1}}
\newcommand*{\Stream}[1]{\ensuremath{\mathrm{Stream}~#1}}
\newcommand*{\Ctx}[2]{\ensuremath{#1,\, #2}}
\newcommand*{\CtxE}[3]{\ensuremath{#1,\, #2 ∶ #3}}
\newcommand*{\All}[2]{\ensuremath{∀ #1.\; #2}}
\newcommand*{\AllE}[3]{\ensuremath{∀ #1 < #2.\; #3}}

\newcommand*{\ap}{\ensuremath{\;}}
\newcommand*{\fun}[2]{\ensuremath{\mathop{λ} #1.\; #2}}

\newcommand*{\lam}[2]{\fun{#1}{#2}}
\newcommand*{\lamE}[3]{\fun{#1 ∶ #2}{#3}}
\newcommand*{\app}[2]{\ensuremath{#1 \ap #2}}
\newcommand*{\slam}[2]{\fun{#1}{#2}}
\newcommand*{\slamE}[3]{\fun{#1 < #2}{#3}}
\newcommand*{\sapp}[2]{\ensuremath{#1 \ap #2}}
\newcommand*{\zero}[1]{\ensuremath{\mathrm{zero} \ap #1}}
\newcommand*{\suc}[3]{\ensuremath{\mathrm{suc} \ap #1 \ap #2 \ap #3}}
\newcommand*{\case}[5]{\ensuremath{\mathrm{caseNat}[#1] \ap #2 \ap #3 \ap #4
    \ap #5}}
\newcommand*{\cons}[3]{\ensuremath{\mathrm{cons} \ap #1 \ap #2 \ap #3}}
\newcommand*{\head}[2]{\ensuremath{\mathrm{head} \ap #1 \ap #2}}
\newcommand*{\tail}[3]{\ensuremath{\mathrm{tail} \ap #1 \ap #2 \ap #3}}
\newcommand*{\fix}[3]{\ensuremath{\mathrm{fix}[#1] \ap #2 \ap #3}}

\newcommand*{\V}[1]{\ensuremath{\mathit{#1}}}

\DeclareMathOperator{\Type}{Type}
\DeclareMathOperator{\refl}{refl}
\DeclareMathOperator{\id}{id}
\DeclareMathOperator{\PRGraphs}{PRGraphs}
\DeclareMathOperator{\PRGraphFam}{PRGraphFam}
\DeclareMathOperator{\PRGraphFams}{PRGraphFams}
\DeclareMathOperator{\Ap}{Ap}
\DeclareMathOperator{\curry}{curry}
\DeclareMathOperator{\eval}{eval}
\DeclareMathOperator{\Size}{Size}
\DeclareMathOperator{\wk}{wk}
\DeclareMathOperator{\Weaken}{Weaken}
\DeclareMathOperator{\Lift}{Lift}
\DeclareMathOperator{\Wk}{Wk}
\DeclareMathOperator{\Id}{Id}
\DeclareMathOperator{\Skip}{Skip}
\DeclareMathOperator{\Fill}{Fill}
\DeclareMathOperator{\mssuc}{ssuc}
\DeclareMathOperator{\Param}{Param}
\DeclareMathOperator{\Cats}{Cats}
\DeclareMathOperator{\Sets}{Sets}
\DeclareMathOperator{\Sizes}{Sizes}

\newcommand*{\Op}[1]{#1^{\mathord{\mathrm{op}}}}


\begin{document}

\title[A Reflexive Graph Model of Sized Types]
  {A Reflexive Graph Model of Sized Types \\ (M.Sc. thesis)}
\author{Jannis Limperg}
\date{28 November 2019}
\institute{Chalmers University Göteborg}
\maketitle

\begin{frame}
  \tableofcontents
\end{frame}


\AtBeginSection[]{
  \begin{frame}
    \vfill
    \begin{beamercolorbox}{title}
      \centering
      \usebeamerfont{title}
      \insertsectionhead
    \end{beamercolorbox}
    \vfill
  \end{frame}
}


\section{Sized Types}

% \begin{frame}[fragile]
%   \frametitle{Structural recursion}

% \begin{code}
% data ℕ : Set where
%   zero : ℕ
%   suc : ℕ → ℕ

% half : ℕ → ℕ
% half zero = zero
% half (suc zero) = zero
% half (suc (suc i)) = suc (half i)
% \end{code}
% \end{frame}


\begin{frame}[fragile]
  \frametitle{Size recursion}

\begin{code}
data ℕₛ (n : Size) : Set where
  zero :                        ℕₛ n
  suc  : (m : Size< n) → ℕₛ m → ℕₛ n

halfₛ : ∀ n → ℕₛ n → ℕₛ n
halfₛ n zero              = zero
halfₛ n (suc m zero)      = zero
halfₛ n (suc m (suc k i)) = suc k (halfₛ k i)
\end{code}
\end{frame}


% \begin{frame}
%   \frametitle{Evaluation of sized types}
%   \begin{block}{Advantages}
%     \begin{itemize}
%       \item Modularity
%       \item Natural support for coinduction
%       \item Easy to implement (in principle)
%     \end{itemize}
%   \end{block}

%   \begin{block}{Disadvantages}
%     \begin{itemize}
%       \item Currently inconsistent in Agda
%       \item Require changes to data types
%     \end{itemize}
%   \end{block}
% \end{frame}


\section{Object Language: λST}

\begin{frame}
  \frametitle{Sizes}

  \begin{Align*}
    n & \bnfdef & x \\
      & |       & 0 \\
      & |       & \ssuc{n} \\
      & |       & ∞
  \end{Align*}

  \pause

  \begin{displaymath}
    0 < \ssuc{0} < \dots < ∞ < \ssuc{∞} < \dots
  \end{displaymath}

  \pause

  \begin{displaymath}
    ⋆ ≔ \ssuc{∞}
  \end{displaymath}
\end{frame}


\begin{frame}
  \frametitle{Types}

  \begin{Align*}
    T, U
      & \bnfdef & \Nat{n} \\
      & | & (\Stream{n}) \\
      & | & T → U \\
      & | & \AllE{x}{n}{T} \\
    \\
    Δ, Ω
      & \bnfdef & () \\
      & | & \ctxE{Δ}{x}{n} \\
    \\
    Γ, Ψ
      & \bnfdef & () \\
      & | & \CtxE{Γ}{x}{T}
  \end{Align*}
\end{frame}


\begin{frame}
  \frametitle{Terms I}

  \begin{Align*}
    t, u, z, s
      & \bnfdef & x \\
      & | & \lamE{x}{T}{t} \\
      & | & \app{t}{u} \\
      \\
      & | & \slamE{x}{n}{t} \\
      & | & \sapp{t}{n}
  \end{Align*}
\end{frame}


\begin{frame}
  \frametitle{Terms II}

  \begin{Align*}
    \mathrm{zero} &∶& \AllE{n}{⋆}{\Nat{n}} \\
    \mathrm{suc} &∶& \AllE{n}{⋆}{\AllE{m}{n}{\Nat{m} → \Nat{n}}} \\
    \mathrm{caseNat}[T] &∶& \AllE{n}{⋆}{\Nat{n} → T → (\AllE{m}{n}{\Nat{m} → T}) → T} \\
    \\
    \pause
    \mathrm{fix}[T(∙)]
      &:& (\AllE{n}{⋆}{(\AllE{m}{n}{T(m)}) → T(n)}) → \AllE{n}{⋆}{T(n)}
  \end{Align*}
\end{frame}


\begin{frame}[fragile]
  \frametitle{Example: \texttt{half}}

\begin{code}
halfₛ : ∀ n → ℕₛ n → ℕₛ n
halfₛ n zero              = zero
halfₛ n (suc m zero)      = zero
halfₛ n (suc m (suc k i)) = suc k (halfₛ k i)
\end{code}

\medskip

\begin{verbatim}
half : ∀ n < ⋆. Nat n → Nat n
half ≔ fix[Nat ∙ → Nat ∙]
  (Λ n < ⋆. λ rec : (∀ m < n. Nat m → Nat m).
    λ i ∶ Nat n. caseNat n i
      (zero n)
      (Λ m < n. λ i′ ∶ Nat m.
        caseNat m i′
          (zero n)
          (Λ k < m. λ i″ ∶ Nat k.
            suc n k (rec k i″))))
\end{verbatim}
\end{frame}


\section{Reflexive Graph Model}

\begin{frame}
  \frametitle{Reflexive graphs}

  \begin{block}{Reflexive graph}
    \begin{itemize}
      \item $Δ ∶ \Type$
      \item $≈_Δ : Δ → Δ → \Type$
      \item $≈_Δ$ is reflexive.
    \end{itemize}
  \end{block}

  \pause

  \begin{block}{Propositional reflexive graph (PRGraph)}
    \begin{itemize}
      \item $Δ$ is a set.
      \item $≈_Δ$ is a proposition.
    \end{itemize}
  \end{block}

  \pause

  \begin{block}{PRGraph morphism from $Δ$ to $Ω$}
    \begin{itemize}
      \item $f ∶ Δ → Ω$
      \item $∀ δ, δ′ ∶ Δ.\; δ ≈_Δ δ′ ⇒ f(δ) ≈_Ω f(δ′)$
    \end{itemize}
  \end{block}
\end{frame}


\begin{frame}
  \frametitle{PRGraph of semantic sizes}
  \begin{Align*}
    \Size &≔& ℕ ⊎ \{ ω + i \mid i ∈ ℕ \} \quad \text{(ordinals below ω·2)}\\
    n ≈_{\Size} m &≔& ⊤
  \end{Align*}
\end{frame}


\begin{frame}
  \frametitle{Interpretation of size contexts and sizes}

  \begin{Align*}
    ⟦Δ⟧ &:& \mathrm{PRGraph} \\
    ⟦()⟧ &≔& ⊤ \\
    ⟦\ctxE{Δ}{x}{n}⟧ &≔& \Sigma_{δ ∶ ⟦Δ⟧} \Size_{< ⟦n⟧(δ)} \\
    \\
    ≈_{⟦Δ⟧}(δ, δ′) &≔& ⊤ \\
    \\
    ⟦n⟧ &:& ⟦Δ⟧ → \Size \\
    \pause
    ⟦x⟧(δ) &≔& \text{(lookup $x$ in $δ$)} \\
    ⟦0⟧(δ) &≔& 0 \\
    ⟦\ssuc{n}⟧(δ) &≔& ⟦n⟧(δ) + 1 \\
    ⟦∞⟧(δ) &≔& ω \\
  \end{Align*}
\end{frame}


\begin{frame}
  \frametitle{Families of PRGraphs}

  Assume a PRGraph $Δ$.
  \begin{block}{Family of PRGraphs over $Δ$}
    \begin{itemize}
      \item $T : Δ → \Type$
      \item $≈_{T,p} : T(δ) → T(δ′) → \Type$ (where $p ∶ δ ≈_Δ δ′$)
      \item $≈_{T,p}$ is reflexive for any $p ∶ δ ≈_Δ δ$
    \end{itemize}
  \end{block}

  \pause

  \begin{block}{PRGraph family morphism from $T$ to $U$}
    \begin{itemize}
      \item $f_δ ∶ T(δ) → U(δ)$
      \item $x ≈_{T,p} y ⇒ f_δ(x) ≈_{U,p} f_{δ′}(y)$
    \end{itemize}
  \end{block}

  Category of PRGraph families over $Δ$ has products, exponentials and a
  terminal object.
\end{frame}


\begin{frame}
  \frametitle{Interpretation of types I}

  \begin{Align*}
    ⟦T⟧ &∶& \PRGraphFam(⟦Δ⟧) \\
    \\
    ⟦\Nat{n}⟧(δ) &≔& ℕ_{≤⟦n⟧(δ)} \\
    i ≈_{⟦\Nat{n}⟧} j &≔& i = j \\
    \\
    ⟦T → U⟧(δ) &≔& ⟦T⟧(δ) ↝ ⟦U⟧(δ) \\
    \multicolumn{3}{l}{\quad \text{(the exponential of PRGraph families over $Δ$)}}
  \end{Align*}
\end{frame}


\begin{frame}
  \frametitle{Interpretation of types II}

  Let $T$ be a PRGraph family over $⟦\ctx{Δ}{n}⟧ = \Sigma_δ \Size_{<⟦n⟧(δ)}$.
  \begin{block}{Size-parametric function}
    Let $f ∶ (m < ⟦n⟧(δ)) → T(δ, m)$. f is size-parametric if
    \begin{displaymath}
      ∀ m, m′ < ⟦n⟧(δ).\; f(m) ≈_{T} f(m′)
    \end{displaymath}
  \end{block}

  \pause

  \begin{block}{Interpretation of size quantification}
    \begin{align*}
      &⟦\AllE{x}{n}{T}⟧(δ) \\
      &\quad ≔ \{ f ∶ (m < ⟦n⟧(δ)) → ⟦T⟧(δ, m) \mid \text{$f$ is size-parametric}\} \\
      &f ≈_{⟦\AllE{x}{n}{T}⟧} g \\
      &\quad ≔ ∀ m < ⟦n⟧(δ); m′ < ⟦n⟧(δ′).\; f(m) ≈_{T} g(m′)
    \end{align*}
  \end{block}
\end{frame}


\begin{frame}
  \frametitle{Interpretation of contexts}

  \begin{Align*}
    ⟦()⟧ &≔& ⊤ \\
    ⟦\CtxE{Γ}{x}{T}⟧ &≔& ⟦Γ⟧ × ⟦T⟧
  \end{Align*}
\end{frame}


\begin{frame}
  \frametitle{Interpretation of terms}

  \begin{displaymath}
    ⟦Δ; Γ ⊢ t ∶ T⟧ : ⟦Γ⟧ → ⟦T⟧ \quad \text{(a PRGraph family morphism)}
  \end{displaymath}

  \pause

  \begin{block}{Example: size abstraction}
    \begin{mathpar}
      \inferrule{\ctxE{Δ}{x}{n}; Γ ⊢ t ∶ T \\ Δ ⊢ Γ}{Δ; Γ ⊢ \slamE{x}{n}{t} ∶ \AllE{x}{n}{T}}
    \end{mathpar}

    Let $⟦Γ⟧$ a PRGraph family over $⟦Δ⟧$; $⟦T⟧$ a PRGraph family over
    $⟦\ctxE{Δ}{x}{n}⟧$; $⟦t⟧_{(δ, m)} : ⟦Γ⟧(δ) → ⟦T⟧(δ, m)$.
    \begin{displaymath}
      ⟦\slamE{x}{n}{t}⟧_δ(γ, m) ≔ ⟦t⟧_{(δ, m)}(γ)
    \end{displaymath}
  \end{block}
\end{frame}


\begin{frame}
  \frametitle{What's the point?}

  \begin{itemize}
    \item<1-> Model \enquote*{obviously} captures size-irrelevance property
    \item<2-> \dots but concluding syntactic size irrelevance is hard.
  \end{itemize}
\end{frame}


\begin{frame}
  \centering
  kthxbye
\end{frame}

\end{document}